\documentclass[11pt,pdftex,dvipsnames,usenames]{beamer}
\setbeameroption{show notes}
%\usepackage[round]{natbib}
%\usepackage{amsmath}
%\usepackage{amssymb}
\usepackage{graphicx}
\usepackage{hyperref}
\DeclareGraphicsExtensions{.ps,.eps,.pdf,.jpg,.png}
\usefonttheme[onlymath]{serif}
%\usepackage[cmintegrals,cmbraces]{newtxmath}
\usetheme{default}
\usecolortheme{dove}

%\usepackage{cancel}
%\usepackage{tikz}
%\usepackage[english]{babel}
\usepackage{statex2}
\usepackage{verbatim}
\usepackage{color}
\usepackage{courier}
\usepackage{listings}
\usepackage{fancyvrb}
%\usepackage{pgf} %portable graphics format
%\usepackage[autobold]{statex2}
%\usepackage{enumitem}
\mode<presentation>
{
  %\usetheme{Warsaw}
  % or ...
  \setbeamercovered{transparent}
  % or whatever (possibly just delete it)

  \setbeamertemplate{navigation symbols}{}
  \usefonttheme[onlysmall]{structurebold}
  %\usefonttheme{structurebold}
}
\addtobeamertemplate{navigation symbols}{}{%
    \usebeamerfont{footline}%
  \setbeamertemplate{navigation symbols}{}
    \usebeamercolor[fg]{footline}%
    \hspace{1em}%
    \insertframenumber/\inserttotalframenumber
}
% \newcommand*{\BART}{\mathrm{BART}\ }
% \newcommand*{\Wei}[2]{\mathrm{Wei}\wrap[()]{#1, #2}}
% \newcommand*{\HBART}{\mathrm{HBART}\ }
% \newcommand*{\corr}{\mathrm{corr}}
% \newcommand*{\abs}{\mathrm{abs}}
% \newcommand*{\DP}[2]{\mb{\mathrm{DP}}\wrap[()]{\mb{#1,\ #2}}}
%\newcommand*{\EV}[2]{\mb{\mathrm{ExtremeValue}}\wrap[()]{\mb{#1,\ #2}}}
%\newcommand*{\Wei}[2]{\mathrm{Wei}\wrap[()]{#1, #2}}

\lstset{basicstyle=\small\ttfamily}
%\lstset{basicstyle=\footnotesize\ttfamily,breaklines=true}
\lstdefinestyle{customR}{
  % belowcaptionskip=1\baselineskip,
  % breaklines=true,
  frame=LT,
  % xleftmargin=\parindent,
  language=R,
showstringspaces=false,
morekeywords={addmargins,dbConnect,dbGetQuery,Postgres,str},
%emphstyle=\color{RedViolet},
  % showstringspaces=false,
  % basicstyle=\footnotesize\ttfamily,
  keywordstyle=\color{VioletRed},
  commentstyle=\color{BrickRed},
  identifierstyle=\color{black},
  stringstyle=\color{ForestGreen},
}
\lstdefinestyle{customSAS}{
  frame=LT,
  language=SAS,
showstringspaces=false,
morekeywords={array,by,data,datafile,dblib,dbdata,endsas,global,if,in,include,keep,let,libname,merge,options,out,proc,qsas,quit,run,select,set,var,where},
  keywordstyle=\color{VioletRed},
  commentstyle=\color{BrickRed},
  identifierstyle=\color{black},
  stringstyle=\color{ForestGreen},
morecomment=[s]{/*}{*/},
morecomment=[s]{*}{;},
}

\RecustomVerbatimCommand{\VerbatimInput}{VerbatimInput}%
{% fontsize=\footnotesize,
 % %
 % frame=lines,  % top and bottom rule only
 % framesep=2em, % separation between frame and text
 % rulecolor=\color{Gray},
 % %
 % label=\fbox{\color{Black}data.txt},
 % labelposition=topline,
 % %
 % commandchars=\|\(\), % escape character and argument delimiters for
 %                      % commands within the verbatim
 % commentchar=*        % comment character
}

\title{An introduction to the CRDW with SQL and SAS}
%\author{Kristen Osinski, Rodney Sparapani and Bradley Taylor}
\date{M1540: May 12, 2025}
\begin{document}
\bibliographystyle{plainnat}

\titlepage
\boldmath
% 0. Intro

 
\begin{frame}[fragile]\frametitle{\bf\textcolor{blue}{Overview}}
\begin{itemize}
%\item Research with Electronic Health Records (EHR) 
\item The CTSI Clinical Research Data Warehouse (CRDW)
\item An overview of the CRDW Jupyterhub environment 
\item Brief history of SQL and SAS
\item Hands-on experience with Jupyterhub %by thoughtful examples
%\item Short introduction to RASMACRO 
\item Online resources: 
links are  \textcolor{PineGreen}{[green]} in square brackets
\end{itemize}
\end{frame}


\begin{frame}[fragile]\frametitle{\bf\textcolor{blue}{CRDW Data Horizon
and Important Eras}}

\begin{itemize}
\item 1989: North American Association of Central Cancer Registries
  (NAACCR) for Froedtert %Memorial Lutheran Hospital % and Community Memorial
\item 1999 to 2018, November: GE/IDX billing
%\item 2001: NAACCR for St.\ Joseph's West Bend
%\item 2003: Philips IntelliSpace Cardiovascular/Xcelera for echos
\item 2004: EPIC EHR debuts at Froedtert
\item 2005: GE MUSE for EKGs
%\item 2005 to 2007: National Provider Identifiers (NPI) transition
\item 2009: American Recovery and Reinvestment Act mandates {\it meaningful use} of EHR (i.e., not just for billing purposes)
%\item 2011: Affordable Care Act mandates {\it meaningful use} of EHR
\item 2012, May: EPIC EHR Community Memorial Menominee Falls
\item 2013: Philips Xcelera Cardiology for echocardiograms
\item \textcolor{red}{2013, July: EPIC EHR for Community Physicians Clinics} 
\item 2013, September: EPIC EHR St.\ Joseph's West Bend
\item 2015: Elekta/MOSAIQ radiotherapy dosage
\item \textcolor{blue}{2015, October: ICD-10 era begins}
%\item 2020, March: COVID-19 pandemic declared
\end{itemize}

\end{frame}


\begin{frame}[fragile]\frametitle{\bf\textcolor{blue}{Resources}}

\begin{itemize}
\item This presentation, programs, etc.\ are available online\\
\textcolor{PineGreen}{[\href{https://github.com/rsparapa/CTSI/tree/main/Summer25}{at my \texttt{github.com} repository}]}
%\item \textcolor{PineGreen}{[\href{https://ctsi.mcw.edu/about/history/support/berd}
%{Biostatistics, Epidemiology and Research Design (BERD)}]}
%\item \textcolor{PineGreen}{[\href{https://ctsi.mcw.edu/investigator/services/ctsi-mini-grants/biostatistical-consultation}
%{BERD Mini-grants}]}
%\item \textcolor{PineGreen}{[\href{https://www.mcw.edu/departments/biostatistics/biostatistics-consulting-service}{Biostatistics Consulting Service (BCS)}]}
\item \textcolor{PineGreen}{[\href{https://www.i2b2.org}
{i2b2: informatics for integrating biology and the bedside}]}
\item \textcolor{PineGreen}{[\href{https://ctsi.mcw.edu/data-science-informatics/cbmi/crdw/bmi-resource-links}
{CTSI Biomedical Informatics links}]}
\item \textcolor{PineGreen}{[\href{https://share.ctsi.mcw.edu/s/8gj39XFxP9oeGGY}{CTSI Honest Broker Data Dictionary}]}
\item \textcolor{PineGreen}{[\href{https://en.wikipedia.org/wiki/Project_Jupyter}{Project Jupyter}]}
\item CRDW Jupyterhub \textcolor{PineGreen}{[\href{https://jupyter.ctsi.mcw.edu}{https://jupyter.ctsi.mcw.edu}]}
\item \textcolor{PineGreen}{[\href{https://www.cdc.gov/nchs/icd/icd9cm.htm}
{ICD-9 manuals available for download}]}
\item \textcolor{PineGreen}{[\href{https://icd10cmtool.cdc.gov}
{US Centers for Disease Control \& Prevention (CDC)\\ 
ICD-10-CM Browser}]}
\item \textcolor{PineGreen}{[\href{https://www.nber.org/research/data/icd-9-cm-and-icd-10-cm-and-icd-10-pcs-crosswalk-or-general-equivalence-mappings}
{US Centers for Medicare \& Medicaid Services (CMS) \\
(with the CDC) ICD-9 to, and from, ICD-10 crosswalk \\
of General Equivalence Mappings (GEM)}]}
%\item \textcolor{PineGreen}{[\href{https://www.neighborhoodatlas.medicine.wisc.edu}{Area Deprivation Index (ADI): out of favor}]}
\end{itemize}

\end{frame}


\begin{frame}[fragile]\frametitle{\bf\textcolor{blue}{CRDW Tables}}
%\begin{center}
\begin{tabular}{lll}
Table Name \texttt{fh\_hb\_NAME\_jupyter} & 
\textcolor{PineGreen}{[\href{https://share.ctsi.mcw.edu/s/8gj39XFxP9oeGGY}{Title in Documentation}]} \\ \hline
\texttt{demographics}       & ``Patient Demographics'' \\ 
\multicolumn{3}{l}{One record per patient: birth date, gender, 
race/ethnicity, death, etc.} \\ \hline 
\texttt{diagnosis}          & ``Diagnosis (Dx)'' \\
\multicolumn{3}{l}{Combo of EPIC/billing with ICD-9/ICD-10 
diagnosis codes} \\  \hline  
\texttt{diagnostic\_results}& ``Diagnostic Results'' \\
\multicolumn{3}{l}{Combo of mainly EPIC with MOSAIQ (for radiotherapy dosage)} \\  \hline  
\texttt{encounters}       & ``Encounters'' \\ 
\multicolumn{3}{l}{Dates/types of all patient encounters} \\ \hline 
\texttt{mar\_table}         & ``Medications Administered''\\
\multicolumn{3}{l}{EPIC \textcolor{red}{medications given} with Medi-Span pharm class/sub-class} \\   \hline 
\texttt{med\_orders\_table} & ``Medication Orders'' \\
\multicolumn{3}{l}{EPIC prescription orders \textcolor{blue}{(not fills!)} along with Medi-Span} \\  \hline  
\texttt{procs}              & ``Procedures (Px)'' \\
\multicolumn{3}{l}{Combo of EPIC/billing with 
\textcolor{red}{ICD-9/ICD-10} 
and \textcolor{blue}{HCPCS/CPT codes}} \\  \hline  
\texttt{vitals}             & ``Vitals'' \\
\multicolumn{3}{l}{EPIC vital signs such as height/weight, blood pressure,
temp, etc. } \\   \hline 
\end{tabular}
%\end{center}
\end{frame}

\begin{frame}[fragile]\frametitle{\bf\textcolor{blue}{CRDW Tables: ``Froedtert Only'' means adults}}
%\begin{center}
\begin{tabular}{lll}
%\multicolumn{2}{c}{Froedtert Only (Meaning Adults Only)} \\
Table Name  \texttt{fh\_hb\_NAME\_jupyter} & 
\textcolor{PineGreen}{[\href{https://share.ctsi.mcw.edu/s/8gj39XFxP9oeGGY}{Title in Documentation}]} \\ \hline
%\texttt{encounters}       & ``Encounters'' \\ 
%\multicolumn{3}{l}{Dates/types of all patient encounters} \\ \hline 
\texttt{naaccr}       & ``NAACCR Data'' \\ 
\multicolumn{3}{l}{North American Association of Central Cancer Registries} \\ \hline 
\texttt{surgical\_case}       & ``Surgical Case'' \\ 
\multicolumn{3}{l}{Including anesthesia, \texttt{asa\_rating\_c}, surgical service, etc.} \\ \hline
%\multicolumn{2}{l}{These are actual table names: unlike \texttt{fh\_hb\_NAME\_jupyter}} \\
%\multicolumn{2}{l}{\texttt{ekg\_patient\_tracings}, \texttt{ekg\_order},} \\
%\multicolumn{2}{l}{\texttt{ekg\_test\_demographics}, \texttt{ekg\_resting\_ecg\_meas},} \\
%\multicolumn{2}{l}{\texttt{ekg\_qrs\_times\_types}, \texttt{ekg\_waveform},} \\
%\multicolumn{2}{l}{\texttt{ekg\_lead\_data}, \texttt{ekg\_pharma\_data},} \\
%\multicolumn{2}{l}{\texttt{ekg\_meas\_matrix} \& \texttt{ekg\_meas\_matrix\_leads}} \\
\multicolumn{2}{l}{\texttt{echo} \qquad \qquad \qquad \qquad \quad ``Echocardiogram Exam Results''} \\ 
\multicolumn{3}{l}{Echocardiogram data from Philips Xcelera: Appendix B} \\ \hline
\multicolumn{2}{l}{Actual table names starting with \texttt{ekg\_}} \\
\multicolumn{3}{l}{GE Healthcare's MUSE for electrocardiograms: Appendix A} \\ \hline
\end{tabular}

\end{frame}

\begin{comment}
\begin{frame}[fragile]\frametitle{\bf\textcolor{blue}{What is an Electronic Health Record?}}

\begin{itemize}
\item Electronic Medical Record (EMR) and\\ 
Electronic Health Record (EHR)\\
are often used interchangeably
\item However, there is a subtle distinction
\item Technically, the EMR is merely a digital version (whether of
  digital provenance or scanned images) of what used to be a paper
  {\it medical record} (a misnomer anyway since NOT limited to
  merely medicine)
\item Typically, the EMR is a collection of health-related data
\item Doctor/nursing/etc.\ notes: not available in the CRDW
\item Imaging, laboratory results, vital signs, etc. 
\item Whereas the EHR is an information management system like EPIC
  providing convenient access to the EMR along with other ancillary
  digital capabilities such as \textcolor{red}{billing}/scheduling,
  prescription pharmaceutical orders and modern data sources 
including X-ray/CAT/MRI scans,  chemo-/radio-therapy dosage,
  electrocardiograms, echocardiograms, etc.
\end{itemize}

\end{frame}
\end{comment}

\begin{frame}[fragile]\frametitle{\bf\textcolor{blue}{What is an Honest Broker?}}

\begin{itemize}
\item
``A neutral intermediary ... between the individual whose ... data are being studied, and the researcher. The honest broker collects and collates pertinent information ... replaces identifiers with a code, and releases only coded information to the researcher.''
 \textcolor{PineGreen}{[\href{https://www.hhs.gov/ohrp/sachrp-committee/recommendations/2011-october-13-letter-attachment-d/index.html}{US
     Health and Human Services FAQ}]}
\item CTSI Biomedical Informatics is the Honest Broker!
\item The term originated in diplomacy meaning an entity
accepted as impartial by all sides in a negotiation
\item German Chancellor Otto von Bismarck first used the term,\\
  applying it to himself, as an intermediary in negotiations
  between Russia and the Ottoman Empire\\ (Auray-Blais and Patenaude,
  BMC Medical Ethics 2006)
\end{itemize}

\end{frame}

\begin{frame}[fragile]\frametitle{\bf\textcolor{blue}{Honest Broker De-identification}}

\begin{itemize}
\item Jupyterhub data is brought ``up-to-date'' on Wednesday nights
\item HIPAA de-identification provided by the Honest Broker
\item For example, patient names, etc.\ are removed
\item The Medical Record Number (MRN),
  \textcolor{red}{\texttt{patient\_mrn}},\\ is replaced by 
  \textcolor{red}{\texttt{patient\_hash}} which is an encrypted
  key
\item \textcolor{red}{\texttt{patient\_hash}} is unchanging so that
the MRNs could be retrieved if you have IRB approval for identified
data
\item All dates for each patient are de-identified by a\\
\textcolor{blue}{single} random integer from -10 to 10 (with zero excluded)
\item This allows any two date differentials to be calculated
exactly such as the age of a diagnosis with respect to birth date
\item Geographically, we have only state of residence and\\
ZIP code shortened to the first 3 digits 
\item Yet, addresses were geocoded to Census Block Groups (CBG)\\ 
for the corresponding Area Deprivation Index (ADI)\\
however, the ADI is out of favor and will be replaced
\end{itemize}

\end{frame}


\begin{frame}[fragile]\frametitle{\bf\textcolor{blue}{A brief introduction to SQL}}

\begin{itemize}
\item Structured Query Language (SQL) 
\item The syntax/semantics for interacting with\\
relational database management systems
\item Originally developed by IBM: now an industry standard
\item \textcolor{PineGreen}{[\href{https://www.iso.org/standard/63555.html}
{SQL:2016 AKA ISO/IEC 9075:2016}]}
\item \textcolor{PineGreen}{[\href{https://www.tiobe.com/tiobe-index}
{The TIOBE Index of programming language popularity}]}\\ (circa 03/25)
\item SQL is ranked 7
\item First appeared in 1974
\end{itemize}

\end{frame}


\begin{frame}[fragile]\frametitle{\bf\textcolor{blue}{A brief introduction to SAS}}

\begin{itemize}
\item The SAS language is a proprietary for-fee fourth-generation\\ domain-specific environment for data science
\item \textcolor{PineGreen}{[\href{https://SAS.com}{https://SAS.com}]}
\item \textcolor{PineGreen}{[\href{https://support.sas.com/en/documentation.html}{https://support.sas.com/en/documentation.html}]}
\item Convenient naturally vectorized \texttt{DATASTEP} language 
\item You don't buy SAS, you rent it annually
\item The MCW site license goes from June to May
\item SAS is free on the Biostatistics/CAPS Linux cluster
%\item It is free for students and members of Biostatistics/CAPS
%\item For others its cheap, use the IS ticket system to order\\
%\textcolor{PineGreen}{[\href{https://mcwcherwellapp.mcwcorp.net/CherwellPortal}{https://mcwcherwellapp.mcwcorp.net/CherwellPortal}]} \\
%\$60 for SAS on the RCC cluster and for Windows installs  
\item On the TIOBE Index of programming language popularity (circa 03/25)
\item SAS is ranked 25 
\item First appeared in 1972
\item The RASMACRO collection of my GPL SAS macros 
\texttt{/usr/local/sasmacro}
%\textcolor{PineGreen}{[\href{https://github.com/rsparapa/rasmacro}{https://github.com/rsparapa/rasmacro}]}
\end{itemize}

\end{frame}


\begin{frame}\frametitle{Atrial fibrillation (AFIB) and atrial flutter (AFLT)}
\begin{itemize}
\item AFIB is most common arrhythmia seen in clinical practice
\item Cause: fibrillating atria {\it f waves}: 300 to 600 bpm
\item AFLT is a closely related condition but less common
\item Cause: atrial {\it flutter waves}: 250 to 350 bpm
\item Typically distressed patients seen in the ER
\item Where AFIB/AFLT is diagnosed with an ECG
\item AHA forecasts 12M AFIB patients in 2030 
\item AFIB: 5X RR for stroke
\item AFIB: 2X RR for all-cause mortality and cognitive dysfunction
\item AFIB associated with heart failure and sudden death
%\item Ventricular rate during untreated AFIB: 100 to 250 bpm
\item {\it paroxysmal} AFIB: spontaneous remitting within 7 days 
\item {\it persistent} AFIB: continuing for more than 7 days 
\item {\it longstanding persistent} AFIB: for more than 1 year
\end{itemize}
\end{frame}


\begin{frame}\frametitle{Atrial fibrillation (AFIB) and atrial flutter (AFLT)}
\begin{itemize}
\item How to assemble an AFIB/AFLT cohort from the CRDW?
\item We need to identify the index event and its ECG
\item ICD-10-CM codes for AFIB: I48
\item Except for the AFLT codes: I48.3, I48.4 and I48.92
\item Ventricular rate during untreated AFIB: 100 to 250 bpm
%\item AFIB: fibrillating atria {\it f waves}: 300 to 600 bpm
%\item AFLT: atrial {\it flutter waves}: 250 to 350 bpm

\item Treatments: ablation, cardioversion, closure and drugs
\item However, atrial pacemakers are NOT effective 
\end{itemize}
\end{frame}

\begin{frame}\frametitle{Electrocardiograms (ECG): Frontal leads}
\begin{center}
\includegraphics[scale=0.45]{.temp/frontal.png}
%\includegraphics[scale=0.6]{.temp/frontal.jpg}
\end{center}
David Strauss and Douglas Schocken\\ 
Marriott’s practical electrocardiography 
%Wolters Kluwer, 13th edition, 2020
\end{frame}

\begin{frame}\frametitle{Electrocardiograms (ECG): Precordial leads}
\begin{center}
\includegraphics[scale=0.45]{.temp/precordial.png}
%\includegraphics[scale=0.6]{.temp/precordial.jpg}
\end{center}
David Strauss and Douglas Schocken\\ 
Marriott’s practical electrocardiography
%Wolters Kluwer, 13th edition, 2020
\end{frame}

\begin{frame}\frametitle{Atrial fibrillation/flutter}
\includegraphics[scale=0.25]{.temp/ECGannotated.png}
%\includegraphics[scale=0.325]{.temp/ECGannotated.png}
\end{frame}

\begin{frame}\frametitle{Left atrial appendage closure}
\begin{center}
\includegraphics[scale=0.25]{.temp/closure.png}
%\includegraphics[scale=0.25]{.temp/closure.png}
\end{center}
Calkins, Tomaselli \& Morady 2012
\end{frame}

\begin{frame}[fragile]
\frametitle{\bf\textcolor{blue}{Hands-on with the CRDW: username and password}}
\begin{center}
\includegraphics[scale=0.2]{.temp/username.png}
\end{center}
\end{frame}


\begin{frame}[fragile]
\frametitle{\bf\textcolor{blue}{Hands-on with the CRDW: \texttt{autoexec.sas}}}
\lstinputlisting[style=customSAS]{snippet0.sas}
\end{frame}

\begin{frame}[fragile]
\frametitle{\bf\textcolor{blue}{Hands-on with the CRDW: \texttt{libname} to database}}
\lstinputlisting[style=customSAS]{snippet1.sas}
\end{frame}

\begin{frame}[fragile]
\frametitle{\bf\textcolor{blue}{Hands-on with the CRDW: SAS pass-through SQL}}
\lstinputlisting[style=customSAS]{snippet2.sas}
\end{frame}

\begin{frame}[fragile]
\frametitle{\bf\textcolor{blue}{Hands-on with the CRDW:
Open \texttt{snippet3.sas}}}
\lstinputlisting[style=customSAS]{snippet3.sas}
\end{frame}

\begin{comment}
\begin{frame}[fragile]
\frametitle{\bf\textcolor{blue}{Hands-on with the CRDW:
Open \texttt{snippet4.sas}}}
\lstinputlisting[style=customSAS]{snippet4.sas}
\end{frame}
\end{comment}

\begin{frame}[fragile]\frametitle{\bf\textcolor{blue}{Hands-on with the CRDW}}
\lstinputlisting[style=customSAS]{snippet5.sas}
\end{frame}

\begin{frame}[fragile]\frametitle{\bf\textcolor{blue}{Medi-Span, GPI and RxNorm
Medical Nomenclature}}
\begin{itemize}
\item \textcolor{PineGreen}{[\href{https://www.wolterskluwer.com/en/solutions/medi-span/about/gpi}
{Medi-Span Generic Product Identifier (GPI)}]}
\item The Wolters Kluwer Medi-Span brand database, called the Medispan
  Electronic Drug File, links the GPI code to other prescription drug
  classification codes
\item \textcolor{PineGreen}{[\href{https://www.nlm.nih.gov/research/umls/rxnorm/index.html}{RxNorm}]} is part of Unified Medical Language System (UMLS)
  terminology maintained by the US National Library
  of Medicine (NLM)
\item GPI and RxNorm codes are available on two CRDW tables
\item ``Medication Orders'' for medicinal prescriptions (not fills!): \texttt{fh\_hb\_med\_orders\_table\_jupyter}
\item ``Medications Administered'' for medicine given: \texttt{fh\_hb\_mar\_table\_jupyter} 
\item Example variables of interest
\item \texttt{pharm\_class}: pharmacologic class
\item \texttt{pharm\_subclass}: pharmacologic subclass
\item \texttt{ingredient\_rxcui\_name}: RxNorm Concept Unique Identifier (CUI) name
\end{itemize}
\end{frame}


\begin{frame}[fragile]\frametitle{\bf\textcolor{blue}{Hands-on with the CRDW}}
\begin{itemize}
\item You can update this lookup table of drug nomenclature\\ 
\item RESOURCE INTENSIVE: DON'T DO IT TODAY \texttt{snippet6.sas}
\item But you can see the output right now: \texttt{medispan.csv}
\end{itemize}
\end{frame}


\begin{frame}[fragile]\frametitle{\bf\textcolor{blue}{Hands-on with the CRDW}}
\lstinputlisting[style=customSAS]{snippet6.sas}
\end{frame}


\begin{frame}[fragile]\frametitle{\bf\textcolor{blue}{Hands-on with the CRDW: 
\textcolor{PineGreen}{[\href{https://icd10cmtool.cdc.gov}
{ICD-10-CM Browser}]}}}
\lstinputlisting[style=customSAS]{snippet7.sas}
\end{frame}


\begin{frame}[fragile]\frametitle{\bf\textcolor{blue}{Hands-on with the CRDW: AFIB warehouse}}
\lstinputlisting[style=customSAS]{snippet8.sas}
\end{frame}


\begin{frame}[fragile]\frametitle{\bf\textcolor{blue}{Hands-on with the CRDW: AFLT warehouse}}
\lstinputlisting[style=customSAS]{snippet9.sas}
\end{frame}


\begin{frame}[fragile]\frametitle{\bf\textcolor{blue}{Hands-on with the CRDW: ECGs}}
\lstinputlisting[style=customSAS]{snippet10.sas}
\end{frame}


\begin{frame}[fragile]\frametitle{\bf\textcolor{blue}{Hands-on with the CRDW: ECGs}}
\lstinputlisting[style=customSAS]{snippet11.sas}
\end{frame}


\begin{frame}[fragile]\frametitle{\bf\textcolor{blue}{Hands-on with the CRDW: ECGs}}
\lstinputlisting[style=customSAS]{snippet12.sas}
\end{frame}


\begin{frame}[fragile]\frametitle{\bf\textcolor{blue}{Hands-on with the CRDW: ECGs}}
\lstinputlisting[style=customSAS]{snippet13.sas}
\end{frame}

\end{document}

