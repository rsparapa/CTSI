\documentclass[11pt,pdftex,dvipsnames,usenames]{beamer}
\setbeameroption{show notes}
%\usepackage[round]{natbib}
%\usepackage{amsmath}
%\usepackage{amssymb}
\usepackage{graphicx}
\usepackage{hyperref}
\DeclareGraphicsExtensions{.ps,.eps,.pdf,.jpg,.png}
\usefonttheme[onlymath]{serif}
%\usepackage[cmintegrals,cmbraces]{newtxmath}
\usetheme{default}
\usecolortheme{dove}

%\usepackage{cancel}
%\usepackage{tikz}
%\usepackage[english]{babel}

\usepackage{verbatim}
\usepackage{color}
%\usepackage{pgf} %portable graphics format
%\usepackage[autobold]{statex2}
%\usepackage{enumitem}
\mode<presentation>
{
  %\usetheme{Warsaw}
  % or ...
  \setbeamercovered{transparent}
  % or whatever (possibly just delete it)

  \setbeamertemplate{navigation symbols}{}
  \usefonttheme[onlysmall]{structurebold}
  %\usefonttheme{structurebold}
}
\addtobeamertemplate{navigation symbols}{}{%
    \usebeamerfont{footline}%
  \setbeamertemplate{navigation symbols}{}
    \usebeamercolor[fg]{footline}%
    \hspace{1em}%
    \insertframenumber/\inserttotalframenumber
}
% \newcommand*{\BART}{\mathrm{BART}\ }
% \newcommand*{\Wei}[2]{\mathrm{Wei}\wrap[()]{#1, #2}}
% \newcommand*{\HBART}{\mathrm{HBART}\ }
% \newcommand*{\corr}{\mathrm{corr}}
% \newcommand*{\abs}{\mathrm{abs}}
% \newcommand*{\DP}[2]{\mb{\mathrm{DP}}\wrap[()]{\mb{#1,\ #2}}}
%\newcommand*{\EV}[2]{\mb{\mathrm{ExtremeValue}}\wrap[()]{\mb{#1,\ #2}}}
%\newcommand*{\Wei}[2]{\mathrm{Wei}\wrap[()]{#1, #2}}

\title{An introduction to big data mining of electronic health records with
Jupyterhub, R, SQL and SAS}
\author{Kristen Osinksi, Rodney Sparapani and Bradley Taylor}

\begin{document}
\bibliographystyle{plainnat}

\titlepage
\boldmath
% 0. Intro
\begin{frame}[fragile]\frametitle{\bf\textcolor{blue}{Outline}}

\begin{itemize}
\item Overview
\item Background
\item Timeline
\end{itemize}

\end{frame}


\begin{frame}[fragile]\frametitle{\bf\textcolor{blue}{Overview}}

\begin{itemize}
\item CTSI Clinical Research Data Warehouse (CRDW)
%\item We restrict our attention on the following matters
\item Adult patient records at Froedtert and MCW
\item Jupyterhub access to the CRDW for adults
\item Collaborative Institutional Training Initiative (CITI) training
\item SQL, R and SAS: popular and mature programming languages
\item SQL and R come free with Jupyterhub
\item SAS can be purchased from MCW IS (free for students)
% \item On the TIOBE Index of programming language popularity (circa 04/22)
% \item Structured Query Language (SQL) is number 9 \\ first appeared in 1974
% \item R is number 11 \\
% derivative of S which first appeared in 1976
% \item SAS is number 21 \\
% first appeared in 1972
\end{itemize}

\end{frame}


\begin{frame}[fragile]\frametitle{\bf\textcolor{blue}{Timeline of CRDW data availability}}

\begin{itemize}
\item 1989: North American Association of Central Cancer Registries (NAACCR)
\item 1999 to 2018, November: GE/IDX billing
\item 2001: NAACCR for St.\ Joseph's West Bend
\item 2003: Philips IntelliSpace Cardiovascular/Xcelera for echos
\item 2004: EPIC EHR debuts at Froedtert
\item 2005: GE/MUSE for EKGs
\item 2005 to 2007: National Provider Identifiers (NPI) transition
\item 2012, May: EPIC EHR Community Memorial Menominee Falls
\item 2013, September: EPIC EHR St.\ Joseph's
\item \textcolor{red}{2013, July: EPIC EHR for Community Physicians Clinics} 
\item 2015: Elekta/MOSAIQ radiotherapy dosage
\item \textcolor{blue}{2015, October: ICD-10-CM/ICD-10-PCS era begins}
\item 2020, March: COVID-19 pandemic declared
\end{itemize}

\end{frame}


\begin{frame}[fragile]\frametitle{\bf\textcolor{blue}{Resources}}

\begin{itemize}
\item \href{https://ctsi.mcw.edu/investigator/services/ctsi-mini-grants/biostatistical-consultation}
{Biostatistical Consultation Mini-grants}
\item \href{https://www.mcw.edu/departments/biostatistics/biostatistics-consulting-service}{Biostatistics Consulting Service (BCS)}
\item \href{https://ctri.mcw.edu/resources/bmi-links}{CTSI Biomedical Informatics links}
\item \href{https://www.nber.org/research/data/icd-9-cm-and-icd-10-cm-and-icd-10-pcs-crosswalk-or-general-equivalence-mappings}
{Centers for Medicare and Medicaid Services ICD-9/ICD-10 crosswalk}
\item \href{https://en.wikipedia.org/wiki/Project_Jupyter}{Project Jupyter}
\item CRDW Jupyterhub \href{https://jupyter.ctsi.mcw.edu}{https://jupyter.ctsi.mcw.edu}
\end{itemize}

\end{frame}

\begin{frame}[fragile]\frametitle{\bf\textcolor{blue}{Comma Separated Values}}

\begin{itemize}
\item A data exchange format that goes back to the early 1970's
\item Popularized by spreadsheets in the 80's and 90's
\item What we have today was solidifed by about 2005 
\item Standards include RFCs 4180 and 7111 among others
\item See \href{https://en.wikipedia.org/wiki/Comma-separated_values}
{``Comma-separated values'' on Wikipedia} 
\item Typically, a three-letter file type of ``csv'',
e.g., ``Book1.csv''
\item A text file where each field is separated by commas
\item A missing value is nothing: two consecutive commas
\item Fields with commas are encased in double-quotes
\item Double-quotes are escaped by doubling them (like SAS does)
\item Double-quotes around numbers (or anything) is read as text 
\item Although, a standard there are edge cases that can't be
  {\it automatically} read by common software such as R and/or SAS
\item Furthermore, spreadsheets are very lax: columns can be a mixture of numbers and text that will cause havoc when read
\end{itemize}

\end{frame}


\begin{frame}[fragile]\frametitle{\bf\textcolor{blue}{Comma Separated Values}}

\begin{itemize}
\item If CSV files have so many challenges, then why bother?
\item No other data exchange formats have caught on
\item Alternatives like XML are even worse!
\item With respect to R and SAS, which are the most relevant here,
CSV files fill an important niche
\item Especially since they make the exchange of dates, times
and date-times effectively trivial
\item This is a very important concern since the EHR is rife
with chronological info: so much so that you don't even want to
consider making their transfer tedious in any shape or form
\item And we have decades of experience with CSVs
\item Largely automated functions
\item R functions: \texttt{read.csv()}/\texttt{write.csv()}
\item SAS: \texttt{PROC IMPORT}/\texttt{PROC EXPORT}
\item Semi-automated
\item SAS DATASTEP and RASMACRO: \texttt{\_cimport}/\texttt{\_cexport}\\
plus \texttt{\_crepair} for an edge case
\end{itemize}

\end{frame}


\begin{frame}[fragile]\frametitle{\bf\textcolor{blue}{A brief introduction to R}}

\begin{itemize}
\item The R language is an open-source free software GNU project\\
 purpose-built for data science with {\it objects}
\item \href{https://R-project.org}{https://R-project.org}
\item \href{https://CRAN.R-project.org/manuals.html}{https://CRAN.R-project.org/manuals.html}
\item Naturally vectorized language with convenient objects such as vectors,
matrices and \texttt{date.frame}'s
\item On the TIOBE Index of programming language popularity (circa 04/22)
\item R is number 11 
\item A derivative of S which first appeared in 1976
\item Many free add-on packages
\item 18991 available on the Comprehensive R Archive Network (CRAN) (circa 04/22)
\item \href{https://CRAN.R-project.org}{https://CRAN.R-project.org}
\end{itemize}

\end{frame}

\begin{frame}[fragile]\frametitle{\bf\textcolor{blue}{Accessing the database with R}}

\begin{itemize}
\item Via \texttt{DBI} package and the \texttt{RPostgres} backend from CRAN 
\item Use your MCWCORP credentials to login at 
\href{https://jupyter.ctsi.mcw.edu}{https://jupyter.ctsi.mcw.edu}
\item BUT YOU HAVE A SEPARATE \texttt{JHUSERNAME/JHPASSWORD}
\end{itemize}
\begin{verbatim}
user="JHUSERNAME", 
password="JHPASSWORD",
library(DBI)
library(RPostgres)
db=dbConnect(
    Postgres(), 
    user=user,
    password=password,
    host="localhost", 
    port=5432, 
    dbname="fh_jupyter_hub_hbdb",
    bigint="integer"
)
\end{verbatim}

\end{frame}


\begin{frame}[fragile]\frametitle{\bf\textcolor{blue}{A brief introduction to SAS}}

\begin{itemize}
\item The SAS language is a proprietary for-fee fourth-generation\\ domain-specific environment for data science
\item \href{https://SAS.com}{https://SAS.com}
\item \href{https://support.sas.com/en/documentation.html}{https://support.sas.com/en/documentation.html}
\item Convenient naturally vectorized \texttt{DATASTEP} language 
\item You don't buy SAS, you rent it annually June to May
\item MCW has a site-license: for students it is free
\item Use the IS ticket system to order
\item On the TIOBE Index of programming language popularity (circa 04/22)
\item SAS is number 21 
\item First appeared in 1972
\item The \texttt{RASMACRO} collection of my GPL SAS macros 
\href{https://github.com/rsparapa/rasmacro}{https://github.com/rsparapa/rasmacro}
\end{itemize}

\end{frame}


\begin{frame}[fragile]\frametitle{\bf\textcolor{blue}{A brief
introduction to SAS}}

\begin{itemize}
\item Why not just use R instead of SAS?
\item R's strengths are not within data processing
\item For example, R does not have warnings about non-unique keys
whereas SAS does (always check your \texttt{.log} !)
\item What is a unique key?
\item Suppose your SSN is unique, does that make it a unique key?
\item NO, not necessarily
\item Consider a table consisting of everyone's lifetime annual
  earnings where each row/record is a year for a given SSN?
\item What is the unique key on this table?
\item It is SSN AND year: NOT SSN alone!
\item The R function \texttt{byvalue} adds SAS-like
automatic variables but it is not the same as the real
thing with WARNINGS
\item For virtually all tables in the CRDW, the unique key
is not obvious and often surprising: SAS is much better suited to this
\item You have been forewarned
\end{itemize}

\end{frame}


\begin{frame}[fragile]\frametitle{\bf\textcolor{blue}{RASMACRO}}
\begin{itemize}
\item SAS has a powerful macro language
\item RASMACRO is GPL library at 
\href{https://github.com/rsparapa/rasmacro}{https://github.com/rsparapa/rasmacro}
\item It is available on the Biostatistics Cheese Cluster
and the Research Computing Center cluster
\item But anyone can install it
\item Very useful to working with CRDW data 
\item \texttt{\_verify} to automatically convert {\it character} fields to numeric if possible,
i.e., some fields are unnecessarily marked as text
\item \texttt{\_constant} automatically drops variables that are constant including missing
\item \texttt{\_crepair} for CSVs with very long variable strings, R has an esoteric format
that needs to be re-formatted so SAS can read them automatically
\end{itemize}
\end{frame}


\begin{frame}[fragile]\frametitle{\bf\textcolor{blue}{RASMACRO}}
\begin{itemize}
\item Installing RASMACRO is fairly trivial on Windows and Linux
\item Get it from github in a directory called \texttt{rasmacro}
\end{itemize}
\texttt{git clone https://github.com/rsparapa/rasmacro.git} \\
To get updates: \texttt{cd rasmacro; git pull} \\
Add these lines to your \texttt{sasv9\_local.cfg} \\
where \texttt{RASMACRO} is the path to your directory 
\begin{verbatim}
/* ' these are single quotes as LaTeX renders them */
-sasautos ('!SASROOT/sasautos' 'RASMACRO') 
-set SASAUTOS ('!SASROOT/sasautos' 'RASMACRO')
\end{verbatim}
\end{frame}

\begin{frame}[fragile]\frametitle{\bf\textcolor{blue}{Hands-on with Jupyterhub}}
\begin{itemize}
\item Login \href{https://jupyter.ctsi.mcw.edu}{https://jupyter.ctsi.mcw.edu}
with your MCWCORP credentials, i.e., same as Outlook/etc.
\item And use your \texttt{JHUSERNAME/JHPASSWORD}
\item Shift+Enter submits the block
\end{itemize}
\begin{verbatim}
library(DBI)
library(RPostgres)
user="JHUSERNAME" 
password="JHPASSWORD"
## db object to facilitate database two-way communication
db=dbConnect(
    Postgres(), ## connect to a PostgreSQL database
    user=user,
    password=password,
    host="localhost", 
    port=5432, 
    dbname="fh_jupyter_hub_hbdb",
    bigint="integer" ## see dbConnect documentation
)
\end{verbatim}
\end{frame}

\begin{frame}[fragile]\frametitle{\bf\textcolor{blue}{Hands-on with Jupyterhub}}
\begin{itemize}
\item Which version of PostgreSQL are we running?
\item At the time of this writing, I see version \textcolor{red}{11}.15
\item The documentation can be found at
\href{https://www.postgresql.org/docs/11}
{https://www.postgresql.org/docs/\textcolor{red}{11}}
\item N.B. \textcolor{red}{11} stands for the major version number, e.g.,
if the software is upgraded to a newer version, then update accordingly
\end{itemize}
\begin{verbatim}
dbGetQuery(db, "select version()")
\end{verbatim}
\texttt{PostgreSQL \textcolor{red}{11}.15 ...}
\end{frame}

\begin{frame}[fragile]\frametitle{\bf\textcolor{blue}{Hands-on with Jupyterhub}}
\begin{itemize}
\item Let's see the public database tables
\end{itemize}
\begin{verbatim}
## ' these are single quotes as LaTeX renders them
tables=dbGetQuery(db, 
       paste("select * from information_schema.tables",
             "where table_schema = 'public'")
)
print(tables$table_name)
\end{verbatim}
\end{frame}

\begin{frame}[fragile]\frametitle{\bf\textcolor{blue}{Hands-on with Jupyterhub}}
\begin{itemize}
\item Let's see the public database table columns
\end{itemize}
\begin{verbatim}
columns=dbGetQuery(db, 
       paste("select * from information_schema.columns",
             "where table_schema = 'public'")
)
str(columns)
table(columns$table_name)
\end{verbatim}
\end{frame}

\begin{frame}[fragile]\frametitle{\bf\textcolor{blue}{Hands-on with Jupyterhub}}
\begin{itemize}
\item We have created the \texttt{columns} \texttt{data.frame}
\item We used the \texttt{str} function to see its {\it structure}: \texttt{str(columns)}
\item The first four variables look like so
\end{itemize}
\begin{verbatim}
'data.frame':	671 obs. of  44 variables:
$ table_catalog : chr "fh_jupyter_hub_hbdb" ...
$ table_schema  : chr "public" "public" "public" ...
$ table_name    : chr "fh_hb_diagnosis_jupyter" ...
$ column_name   : chr "patient_hash" "encounter_hash" ...
...
\end{verbatim}
\end{frame}

\begin{frame}[fragile]\frametitle{\bf\textcolor{blue}{Hands-on with Jupyterhub}}
\begin{itemize}
\item We have quickly summarized the tables in the database
\item \texttt{table(columns\$table\_name)}
\item The counts correspond to the number of columns as we have seen from the structure of the \texttt{data.frame}
%\item Here's an edited version focusing on the most important tables
\end{itemize}
\begin{verbatim}
      fh_hb_demographics_jupyter          
                              31
         fh_hb_diagnosis_jupyter 
                              15 
fh_hb_diagnostic_results_jupyter
                              27
         fh_hb_mar_table_jupyter
                              37
  fh_hb_med_orders_table_jupyter 
                              32 
             fh_hb_procs_jupyter
                              18 
            fh_hb_vitals_jupyter 
                              11 
\end{verbatim}
\end{frame}

\begin{frame}[fragile]\frametitle{\bf\textcolor{blue}{Hands-on with Jupyterhub}}
\begin{itemize}
\item When does NAACCR data start?
\item Above, I said 1989: how did I get that?
\item N.B. The SQL \texttt{date\_part} function is not to be confused with 
the SAS \texttt{datepart} function
\end{itemize}
\begin{verbatim}
dxyear=dbGetQuery(db, 
    paste("select",
          "date_part('year', date_of_diagnosis_shifted)",
          "from fh_hb_naaccr_jupyter")
)
str(dxyear)
t(table(dxyear$date_part))
\end{verbatim}
\end{frame}

\begin{frame}[fragile]\frametitle{\bf\textcolor{blue}{Hands-on with Jupyterhub}}
\begin{itemize}
\item When does EKG data start?
\end{itemize}
\begin{verbatim}
ekgyear=dbGetQuery(db, 
    paste("select",
          "date_part('year',acquisition_date_shifted)",
          "from ekg_test_demographics")
)
str(ekgyear)
t(table(ekgyear$date_part))
\end{verbatim}
\end{frame}

\end{document}

