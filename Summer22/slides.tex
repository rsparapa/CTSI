\documentclass[11pt,pdftex,dvipsnames,usenames]{beamer}
\setbeameroption{show notes}
%\usepackage[round]{natbib}
%\usepackage{amsmath}
%\usepackage{amssymb}
\usepackage{graphicx}
\usepackage{hyperref}
\DeclareGraphicsExtensions{.ps,.eps,.pdf,.jpg,.png}
\usefonttheme[onlymath]{serif}
%\usepackage[cmintegrals,cmbraces]{newtxmath}
\usetheme{default}
\usecolortheme{dove}

%\usepackage{cancel}
%\usepackage{tikz}
%\usepackage[english]{babel}
\usepackage{statex2}
\usepackage{verbatim}
\usepackage{color}
\usepackage{courier}
\usepackage{listings}
\usepackage{fancyvrb}
%\usepackage{pgf} %portable graphics format
%\usepackage[autobold]{statex2}
%\usepackage{enumitem}
\mode<presentation>
{
  %\usetheme{Warsaw}
  % or ...
  \setbeamercovered{transparent}
  % or whatever (possibly just delete it)

  \setbeamertemplate{navigation symbols}{}
  \usefonttheme[onlysmall]{structurebold}
  %\usefonttheme{structurebold}
}
\addtobeamertemplate{navigation symbols}{}{%
    \usebeamerfont{footline}%
  \setbeamertemplate{navigation symbols}{}
    \usebeamercolor[fg]{footline}%
    \hspace{1em}%
    \insertframenumber/\inserttotalframenumber
}
% \newcommand*{\BART}{\mathrm{BART}\ }
% \newcommand*{\Wei}[2]{\mathrm{Wei}\wrap[()]{#1, #2}}
% \newcommand*{\HBART}{\mathrm{HBART}\ }
% \newcommand*{\corr}{\mathrm{corr}}
% \newcommand*{\abs}{\mathrm{abs}}
% \newcommand*{\DP}[2]{\mb{\mathrm{DP}}\wrap[()]{\mb{#1,\ #2}}}
%\newcommand*{\EV}[2]{\mb{\mathrm{ExtremeValue}}\wrap[()]{\mb{#1,\ #2}}}
%\newcommand*{\Wei}[2]{\mathrm{Wei}\wrap[()]{#1, #2}}

\lstset{basicstyle=\small\ttfamily}
%\lstset{basicstyle=\footnotesize\ttfamily,breaklines=true}
\lstdefinestyle{customR}{
  % belowcaptionskip=1\baselineskip,
  % breaklines=true,
  frame=LT,
  % xleftmargin=\parindent,
  language=R,
showstringspaces=false,
morekeywords={addmargins,dbConnect,dbGetQuery,Postgres,str},
%emphstyle=\color{RedViolet},
  % showstringspaces=false,
  % basicstyle=\footnotesize\ttfamily,
  keywordstyle=\color{VioletRed},
  commentstyle=\color{BrickRed},
  identifierstyle=\color{black},
  stringstyle=\color{ForestGreen},
}
\lstdefinestyle{customSAS}{
  frame=LT,
  language=SAS,
showstringspaces=false,
morekeywords={proc,data,datafile,out,run},
  keywordstyle=\color{VioletRed},
  commentstyle=\color{BrickRed},
  identifierstyle=\color{black},
  stringstyle=\color{ForestGreen},
}

\RecustomVerbatimCommand{\VerbatimInput}{VerbatimInput}%
{% fontsize=\footnotesize,
 % %
 % frame=lines,  % top and bottom rule only
 % framesep=2em, % separation between frame and text
 % rulecolor=\color{Gray},
 % %
 % label=\fbox{\color{Black}data.txt},
 % labelposition=topline,
 % %
 % commandchars=\|\(\), % escape character and argument delimiters for
 %                      % commands within the verbatim
 % commentchar=*        % comment character
}

\title{An introduction to big data mining of electronic health records with
Jupyterhub, R, SQL and SAS}
\author{Kristen Osinski, Rodney Sparapani and Bradley Taylor}
\date{H1210/H1230/H1250: July 15, 2022}
\begin{document}
\bibliographystyle{plainnat}

\titlepage
\boldmath
% 0. Intro

\begin{frame}[fragile]\frametitle{\bf\textcolor{blue}{Abstract}}
  Since the \emph{American Recovery and Reinvestment Act} mandated
  meaningful usage of Electronic Health Records (EHR) data systems in
  2009, there has been a paradigm shift in clinical/translational
  research towards studying EHR administrative databanks.  The
  Clinical and Translational Science Institute (CTSI) of Southeast
  Wisconsin has been at the forefront of this emerging trend.  CTSI
  Biomedical Informatics has painstakingly created the Clinical
  Research Data Warehouse (CRDW) with its Honest Broker tables.  The
  CRDW provides access to the EPIC EHR and other ancillary digital
  databanks. However, searching and manipulating the CRDW was
  challenging even for those familiar with it.  Recently, in 2020, a
  Jupyterhub server was created so that users could explore the CRDW
  directly with relative ease via programming languages in common use
  among biostatisticians, bioinformaticians and data scientists: SQL,
  R and SAS.  This workshop will provide you with a hands-on
  introduction to the CRDW via Jupyterhub.  CITI training and MCWCORP
  username/password authentication is recommended.  Room capacity
  limit: 90.
\end{frame}
 
\begin{frame}[fragile]\frametitle{\bf\textcolor{blue}{Learning
      Objectives}}
\begin{itemize}
\item Definitions/jargon for Electronic Health Records (EHR)
\item A series of online resources are developed for EHR
\item The CTSI Clinical Research Data Warehouse (CRDW)
\item An overview of the CRDW Jupyterhub environment 
\item Brief history of SQL, R and SAS
\item Hands-on experience with Jupyterhub by thoughtful examples
\item Short introduction of RASMACRO 
\end{itemize}
\end{frame}

\begin{frame}[fragile]\frametitle{\bf\textcolor{blue}{Schedule}}

\begin{tabular}{rl}
8:00 & Speaker/attendee introductions \\
8:15 & Bradley Taylor, Chief Research Informatics Officer \\
8:30 & Kristen Osinski, Business Analyst IV, CTSI \\
9:15 & Overview: slides 0:20  \\
10:00& Snack break: getting to know each other \\
10:30& Hands-on with Jupyterhub: slides 21:32  \\
12:00& Lunch provided \\
12:30& Hands-on with Jupyterhub: slides 33:43  \\
2:00 & Biological break \\
2:15 & Hands-on with Jupyterhub: slides 44:55 \\
3:45 & Q and A \\
4:15 & Closing ceremonies
\end{tabular}

\end{frame}

\begin{comment}
\begin{frame}[fragile]\frametitle{\bf\textcolor{blue}{Outline}}

\begin{itemize}
\item Overview
\item Background
\item Timeline
\end{itemize}

\end{frame}
\end{comment}

\begin{frame}[fragile]\frametitle{\bf\textcolor{blue}{Overview}}

\begin{itemize}
\item Clinical and Translational Science Institute\\
of Southeast Wisconsin (CTSI)
\item CTSI Clinical Research Data Warehouse (CRDW)
%\item We restrict our attention on the following matters
\item CRDW contains adult electronic health records (EHR)\\
at Froedtert and MCW: mainly EPIC EHR\\
but including billing claims and ancillary databanks
%including the EPIC EHR, billing and other sources of data
\item Jupyterhub access to the CRDW for adults
\item Collaborative Institutional Training Initiative (CITI) required
\item SQL, R and SAS: popular and mature programming languages
\item SQL and R come free with Jupyterhub
\item SAS can be purchased through the MCW IS Ticket system\\
 for Linux server and Windows clients, it's either free or cheap
%\$60 for Linux on the RCC cluster and for Windows installs  
\item \textcolor{PineGreen}{[\href{https://mcwcherwellapp.mcwcorp.net/CherwellPortal}{https://mcwcherwellapp.mcwcorp.net/CherwellPortal}]}
\item In this presentation, links are colored MCW green \textcolor{PineGreen}{[green]}
and surrounded by square brackets for color equity

% \item On the TIOBE Index of programming language popularity (circa 04/22)
% \item Structured Query Language (SQL) is number 9 \\ first appeared in 1974
% \item R is number 11 \\
% derivative of S which first appeared in 1976
% \item SAS is number 21 \\
% first appeared in 1972
\end{itemize}

\end{frame}


\begin{frame}[fragile]\frametitle{\bf\textcolor{blue}{CRDW Data Horizon
and Important Eras}}

\begin{itemize}
\item 1989: North American Association of Central Cancer Registries
  (NAACCR) for Froedtert %Memorial Lutheran Hospital % and Community Memorial
\item 1999 to 2018, November: GE/IDX billing
%\item 2001: NAACCR for St.\ Joseph's West Bend
%\item 2003: Philips IntelliSpace Cardiovascular/Xcelera for echos
\item 2004: EPIC EHR debuts at Froedtert
\item 2005: GE/MUSE for EKGs
\item 2005 to 2007: National Provider Identifiers (NPI) transition
\item 2009: American Recovery and Reinvestment Act mandates {\it meaningful use} of EHR (i.e., not just for billing purposes)
%\item 2011: Affordable Care Act mandates {\it meaningful use} of EHR
\item 2012, May: EPIC EHR Community Memorial Menominee Falls
\item \textcolor{red}{2013, July: EPIC EHR for Community Physicians Clinics} 
\item 2013, September: EPIC EHR St.\ Joseph's West Bend
\item 2015: Elekta/MOSAIQ radiotherapy dosage
\item \textcolor{blue}{2015, October: ICD-10 era begins}
\item 2020, March: COVID-19 pandemic declared
\end{itemize}

\end{frame}


\begin{frame}[fragile]\frametitle{\bf\textcolor{blue}{Resources}}

\begin{itemize}
\item This presentation, programs, etc.\ are available online\\
\textcolor{PineGreen}{[\href{https://github.com/rsparapa/CTSI/tree/main/Summer22}{at my \texttt{github.com} repository}]}
\item \textcolor{PineGreen}{[\href{https://ctsi.mcw.edu/about/history/support/berd}
{Biostatistics, Epidemiology and Research Design (BERD)}]}
\item \textcolor{PineGreen}{[\href{https://ctsi.mcw.edu/investigator/services/ctsi-mini-grants/biostatistical-consultation}
{BERD Mini-grants}]}
\item \textcolor{PineGreen}{[\href{https://www.mcw.edu/departments/biostatistics/biostatistics-consulting-service}{Biostatistics Consulting Service (BCS)}]}
\item \textcolor{PineGreen}{[\href{https://www.i2b2.org}
{i2b2: informatics for integrating biology and the bedside}]}
\item \textcolor{PineGreen}{[\href{https://ctri.mcw.edu/resources/bmi-links}
%{Clinical and Translational Science Institute\\ 
%of Southeast Wisconsin (CTSI)\\ 
{CTSI Biomedical Informatics links}]}
\item \textcolor{PineGreen}{[\href{https://ctri.mcw.edu/wp-content/uploads/CTSI-Honest-Broker-Data-Dictionary.pdf}{CTSI Honest Broker Data Dictionary}]}
\item \textcolor{PineGreen}{[\href{https://en.wikipedia.org/wiki/Project_Jupyter}{Project Jupyter}]}
\item CRDW Jupyterhub \textcolor{PineGreen}{[\href{https://jupyter.ctsi.mcw.edu}{https://jupyter.ctsi.mcw.edu}]}
\item \textcolor{PineGreen}{[\href{https://www.cdc.gov/nchs/icd/icd9cm.htm}
{ICD-9 manuals available for download}]}
\item \textcolor{PineGreen}{[\href{https://icd10cmtool.cdc.gov}
{US Centers for Disease Control \& Prevention (CDC)\\ 
ICD-10-CM Browser}]}
\item \textcolor{PineGreen}{[\href{https://www.nber.org/research/data/icd-9-cm-and-icd-10-cm-and-icd-10-pcs-crosswalk-or-general-equivalence-mappings}
{US Centers for Medicare \& Medicaid Services (CMS) \\
(with the CDC) ICD-9 to, and from, ICD-10 crosswalk \\
of General Equivalence Mappings (GEM)}]}
\item \textcolor{PineGreen}{[\href{https://www.neighborhoodatlas.medicine.wisc.edu}{Area Deprivation Index (ADI)}]}
\end{itemize}

\end{frame}

\begin{frame}[fragile]\frametitle{\bf\textcolor{blue}{What is an Electronic Health Record?}}

\begin{itemize}
\item Electronic Medical Record (EMR) and\\ 
Electronic Health Record (EHR)\\
are often used interchangeably
\item However, there is a subtle distinction
\item Technically, the EMR is merely a digital version (whether of
  digital provenance or scanned images) of what used to be a paper
  {\it medical record} (which is a misnomer, i.e., NOT limited to
  merely medicine as the name seemingly implies)
\item Typically, the EMR is a collection of health-related data
\item Inpatient and outpatient doctor/nursing/etc.\ treatment notes
\item Imaging, laboratory results, vital signs, etc. 
\item Whereas the EHR is an information management system like EPIC
  providing convenient access to the EMR along with other ancillary
  digital capabilities such as \textcolor{red}{billing}/scheduling,
  prescription pharmaceutical orders and modern data sources 
including X-ray/CAT/MRI scans,  chemo-/radio-therapy dosage,
  electrocardiograms, echocardiograms, etc.
\end{itemize}

\end{frame}


\begin{frame}[fragile]\frametitle{\bf\textcolor{blue}{What is an Honest Broker?}}

\begin{itemize}
\item
``A neutral intermediary ... between the individual whose ... data are being studied, and the researcher. The honest broker collects and collates pertinent information ... replaces identifiers with a code, and releases only coded information to the researcher.''
 \textcolor{PineGreen}{[\href{https://www.hhs.gov/ohrp/sachrp-committee/recommendations/2011-october-13-letter-attachment-d/index.html}{US
     Health and Human Services FAQ}]}
\item CTSI Biomedical Informatics is the Honest Broker!
\item The term originated in diplomacy meaning an entity
accepted as impartial by all sides in a negotiation
\item German Chancellor Otto von Bismarck was the first to use the
  term, by applying it to himself, as an intermediary in negotiations
  between Russia and the Ottoman Empire (Auray-Blais and Patenaude,
  BMC Medical Ethics 2006)
\end{itemize}

\end{frame}

\begin{frame}[fragile]\frametitle{\bf\textcolor{blue}{Honest Broker De-identification}}

\begin{itemize}
\item Jupyterhub data is brought ``up-to-date'' on Wednesday nights
\item HIPAA de-identification provided by the Honest Broker
\item For example, patient names, etc.\ are removed
\item The Medical Record Number (MRN),
  \textcolor{red}{\texttt{patient\_mrn}},\\ is replaced by 
  \textcolor{red}{\texttt{patient\_hash}} which is an encrypted
  key
\item \textcolor{red}{\texttt{patient\_hash}} is unchanging so that
the MRNs could be retrieved if you have IRB approval for identified
data
\item All dates for each patient are de-identified by a\\
\textcolor{blue}{single} random integer from -10 to 10 (with zero excluded)
\item This allows any two date differentials to be calculated
exactly such as the age of a diagnosis with respect to birth date
\item Geographically, we have only state of residence and\\
ZIP code shortened to the first 3 digits 
\item Yet, addresses were geocoded to Census Block Groups (CBG)\\ 
and the corresponding Area Deprivation Index is provided\\
(the CBG is not of course)
\end{itemize}

\end{frame}


\begin{frame}[fragile]\frametitle{\bf\textcolor{blue}{A brief introduction to SQL}}

\begin{itemize}
\item Structured Query Language (SQL) 
\item The syntax/semantics for interacting with\\
relational database management systems
\item Originally developed by IBM: now an industry standard
\item \textcolor{PineGreen}{[\href{https://www.iso.org/standard/63555.html}
{SQL:2016 AKA ISO/IEC 9075:2016}]}
\item \textcolor{PineGreen}{[\href{https://www.tiobe.com/tiobe-index}
{The TIOBE Index of programming language popularity}]}\\ (circa 04/22)
\item SQL is ranked 9
\item First appeared in 1974
\end{itemize}

\end{frame}

\begin{frame}[fragile]\frametitle{\bf\textcolor{blue}{A brief introduction to R}}

\begin{itemize}
\item The R language is an open-source, free software GNU project\\
 purpose-built for data science that is {\it object-oriented}
\item \textcolor{PineGreen}{[\href{https://R-project.org}{https://R-project.org}]}
\item \textcolor{PineGreen}{[\href{https://CRAN.R-project.org/manuals.html}{https://CRAN.R-project.org/manuals.html}]}
\item Naturally vectorized language with convenient objects such as vectors,
matrices, lists (objects containers) and 
\textcolor{red}{\texttt{date.frame}}'s
\item On the TIOBE Index of programming language popularity (circa 04/22)
\item R is ranked 11 
\item A derivative of S which first appeared in 1976
\item Many free add-on packages
\item 18991 available on the Comprehensive R Archive Network (CRAN) (circa 04/22) \textcolor{PineGreen}{[\href{https://CRAN.R-project.org}{https://CRAN.R-project.org}]}
\item N.B. Jupyterhub can be used with either R or Python 3\\ 
(Python is ranked 1 on the TIOBE Index circa 04/22)
\end{itemize}

\end{frame}

\begin{frame}[fragile]\frametitle{\bf\textcolor{blue}{Accessing the database with R}}

\begin{itemize}
\item Via \textcolor{red}{\texttt{DBI}} package and the 
\textcolor{red}{\texttt{RPostgres}} backend from CRAN 
\item Use your MCWCORP credentials to log in online at 
\textcolor{PineGreen}{[\href{https://jupyter.ctsi.mcw.edu}{https://jupyter.ctsi.mcw.edu}]}
\item BUT YOU HAVE A SEPARATE \textcolor{blue}{\texttt{JHUSERNAME/JHPASSWORD}}
\end{itemize}
\lstinputlisting[style=customR]{snippet1.R}
%\VerbatimInput{snippet1.R}
% \begin{verbatim}
% user="JHUSERNAME" 
% password="JHPASSWORD"
% library(DBI)
% library(RPostgres)
% db=dbConnect(
%     Postgres(), 
%     dbname  ="fh_jupyter_hub_hbdb",
%     user    =user,
%     password=password,
%     host    ="localhost", 
%     port    =5432, 
%     bigint  ="integer"
% )
% \end{verbatim}

\end{frame}


\begin{frame}[fragile]\frametitle{\bf\textcolor{blue}{A brief introduction to SAS}}

\begin{itemize}
\item The SAS language is a proprietary for-fee fourth-generation\\ domain-specific environment for data science
\item \textcolor{PineGreen}{[\href{https://SAS.com}{https://SAS.com}]}
\item \textcolor{PineGreen}{[\href{https://support.sas.com/en/documentation.html}{https://support.sas.com/en/documentation.html}]}
\item Convenient naturally vectorized \texttt{DATASTEP} language 
\item You don't buy SAS, you rent it annually
\item The MCW site license goes from June to May
\item It is free for students and members of Biostatistics/CAPS
\item For others its cheap, use the IS ticket system to order\\
\textcolor{PineGreen}{[\href{https://mcwcherwellapp.mcwcorp.net/CherwellPortal}{https://mcwcherwellapp.mcwcorp.net/CherwellPortal}]} \\
\$60 for SAS on the RCC cluster and for Windows installs  
\item On the TIOBE Index of programming language popularity (circa 04/22)
\item SAS is ranked 21 
\item First appeared in 1972
\item The RASMACRO collection of my GPL SAS macros 
\textcolor{PineGreen}{[\href{https://github.com/rsparapa/rasmacro}{https://github.com/rsparapa/rasmacro}]}
\end{itemize}

\end{frame}


\begin{frame}[fragile]\frametitle{\bf\textcolor{blue}{A brief
introduction to SAS}}

\begin{itemize}
\item Why not just use R instead of SAS?
\item R's strengths are not within data processing
\item For example, R does not have warnings about non-unique keys
whereas SAS does (always check your \texttt{.log} !)
\item This begs the question: ``What is a unique key?''
\item Suppose your SSN is unique, does that make it a unique key?
\item NO! %, not necessarily
\item Consider a table consisting of annual earnings\\ 
where each row/record is a year for a given SSN?
\item What is the unique key on this table?
\item It is SSN AND year: NOT SSN alone!
\item The R function \texttt{byvalue} adds SAS-like
FIRST./LAST. automatic variables but it is not the same as the real
thing with ERRORS/WARNINGS (SAS provides them, R is silent)
\item For virtually all tables in the CRDW, the unique key
is not obvious and often surprising: SAS is much better suited to this
\item You have been forewarned
\end{itemize}

\end{frame}

\begin{frame}[fragile]\frametitle{\bf\textcolor{blue}{A brief
introduction to Comma Separated Values}}

\begin{itemize}
\item A data exchange format that goes back to the early 1970's
\item Popularized by spreadsheets in the 80's and 90's
\item What we have today was solidifed by about 2005 
\item Standards include RFCs 4180 and 7111 among others
\item See \textcolor{PineGreen}{[\href{https://en.wikipedia.org/wiki/Comma-separated_values}
{``Comma-separated values'' on Wikipedia}]}
\item Typically, a three-letter file type of ``csv'',
e.g., ``Book1.csv''
\item A text file where each field is separated by commas
\item A missing value is nothing: two consecutive commas
\item Fields with commas are encased in double-quotes
\item Double-quotes are escaped by doubling them (like SAS does)
\item Double-quotes around numbers (or anything) is read as text 
\item Although, a standard there are edge cases that can't be
  {\it automatically} read by common software such as R and/or SAS
\item Furthermore, spreadsheets are very lax: columns can be a mixture of numbers and text that will create havoc when read
\end{itemize}

\end{frame}


\begin{frame}[fragile]\frametitle{\bf\textcolor{blue}{A brief
introduction to Comma Separated Values}}

\begin{itemize}
\item If CSV files have so many challenges, then why bother?
\item No other data exchange formats have caught on
\item Alternatives like {\it recfiles} are not nearly as popular
\item With respect to CSV, SAS fills an important niche
\item SAS makes the exchange of CSV files with dates, times
and date-times effectively trivial
\item This is a very important concern since the EHR is rife
with chronological info: so much so that you don't even want to
consider making their transfer manual in any shape or form
\item And we have decades of experience with CSVs
\item Largely automated functions roughly in order of usefulness
\item SAS: \texttt{PROC IMPORT}/\texttt{PROC EXPORT}
\item R functions: \texttt{read.csv()}/\texttt{write.csv()}
\item RASMACRO: \texttt{\_cimport}/\texttt{\_cexport}\\
plus \texttt{\_crepair}, \texttt{\_constant}
and \texttt{\_verify} for handy features
\end{itemize}

\end{frame}

\begin{frame}[fragile]\frametitle{\bf\textcolor{blue}{CSV processing with R}}
\begin{itemize}
\item You can download your CSV files as we will see later
\item Either with your web browser or with Secure Copy: \texttt{scp}
\item \texttt{scp} is a standard command on Windows, macOS and Linux
\item However, \textcolor{red}{due to IRB restrictions, 
you may not download data
for more than 9999 PATIENTS} in one project
\item This is the honor system since there is no mechanism
to prevent malfeasance %(but their could be ``real soon now'')
\item And you can process them with R
\item R will convert character to numeric values where possible
\item Except that it handles dates as character by default!
\item There is no automatic detection: that is a manual process
\end{itemize}
\begin{verbatim}
CMD% scp USER@jupyter.ctsi.mcw.edu:naaccr.csv DIR
R> naaccr = read.csv("DIR/naaccr.csv")
R> str(naaccr)
R> ?str
\end{verbatim}
\end{frame}

\begin{frame}[fragile]\frametitle{\bf\textcolor{blue}{A brief
introduction to RASMACRO}}
\begin{itemize}
\item SAS has a powerful macro language
\item RASMACRO is GPL library at 
\textcolor{PineGreen}{[\href{https://github.com/rsparapa/rasmacro}{https://github.com/rsparapa/rasmacro}]}
\item It is available on the Biostatistics Linux Cluster
and the Research Computing Center cluster
\item But anyone can install it
\item Very useful to working with CRDW data 
\item \texttt{\_verify} to automatically convert {\it character} fields to numeric if possible,
i.e., some fields are unnecessarily encoded as text
\item \texttt{\_constant} automatically drops variables that are constant including missing
\item \texttt{\_crepair} for CSVs with very long variable strings\\
R will create an esoteric CSV file\\
that needs re-formatting so SAS can read it automatically
\end{itemize}
\end{frame}


\begin{frame}[fragile]\frametitle{\bf\textcolor{blue}{A brief
introduction to RASMACRO}}
\begin{itemize}
\item Installing RASMACRO is fairly trivial on Windows and Linux
\item Get it from github in a directory called \texttt{rasmacro}
\end{itemize}
\texttt{git clone https://github.com/rsparapa/rasmacro.git} \\
To get updates: \texttt{cd rasmacro; git pull} \\
Add these lines to your \texttt{sasv9\_local.cfg} \\
where \texttt{RASMACRO} is the path to your directory 
\begin{verbatim}
/* ' these are single quotes */
-sasautos ('!SASROOT/sasautos' 'RASMACRO') 
-set SASAUTOS ('!SASROOT/sasautos' 'RASMACRO')
\end{verbatim}
\end{frame}

\begin{frame}[fragile]\frametitle{\bf\textcolor{blue}{CSV processing with SAS}}
\begin{itemize}
\item SAS does NOT automatically convert character to numeric
\item However, the RASMACRO \texttt{\_verify} can handle that
\item SAS does automatically detect dates, times and datetimes!
\item The RASMACRO \texttt{\_constant} automatically detects
variables that are constant (including missing) and
removes them
\item In this case, the following were superfluous and removed
\item \texttt{mult\_tum\_rpt\_as\_one\_prim, multiplicity\_counter, 
cs\_tumor\_size, cs\_lymph\_nodes, cs\_mets\_at\_dx, cs\_mets\_at\_dx\_bone, 
cs\_mets\_at\_dx\_brain, cs\_mets\_at\_dx\_liver, cs\_mets\_at\_dx\_lung, 
sequence\_number\_hospital}
\end{itemize}
\lstinputlisting[style=customSAS]{snippet1.sas}
% \begin{verbatim}
% proc import datafile="DIR/naaccr.csv" out=naaccr;
% guessingrows=max;
% run;
% %_verify(data=naaccr, out=naaccr);
% %_constant(data=naaccr, out=naaccr);
% \end{verbatim}

\end{frame}

\begin{frame}[fragile]\frametitle{\bf\textcolor{blue}{Hands-on with Jupyterhub}}
\begin{itemize}
\item With your web browser, login
  \textcolor{PineGreen}{[\href{https://jupyter.ctsi.mcw.edu}{https://jupyter.ctsi.mcw.edu}]} \\
using MCWCORP credentials, i.e., same as Outlook/etc.
\item And use your Jupyterhub-only \texttt{JHUSERNAME/JHPASSWORD}
\item Pressing Shift+Enter on your keyboard submits the code
\end{itemize}
\lstinputlisting[style=customR]{snippet2.R}
% \begin{verbatim}
% library(DBI)
% library(RPostgres)
% user="JHUSERNAME" 
% password="JHPASSWORD"
% ## db object to facilitate database two-way communication
% db=dbConnect(
%     Postgres(),           ## connect to PostgreSQL
%     dbname  ="fh_jupyter_hub_hbdb",
%     user    =user,
%     password=password,
%     host    ="localhost", ## loopback web connection 
%     port    =5432, 
%     bigint  ="integer"    ## see dbConnect documentation
% )
% \end{verbatim}
\end{frame}

\begin{frame}[fragile]\frametitle{\bf\textcolor{blue}{Hands-on with Jupyterhub}}
\begin{itemize}
\item Which version of PostgreSQL are we running?
\item At the time of this writing, I see version \textbf{\textcolor{red}{11}}.15
\item The documentation can be found at
\textcolor{PineGreen}{[\href{https://www.postgresql.org/docs/11}
{https://www.postgresql.org/docs/\textbf{\textcolor{red}{11}}}]}
\item N.B. \textbf{\textcolor{red}{11}} stands for the major version number, e.g.,
if the software is upgraded to a newer version, then update accordingly
\end{itemize}
\begin{lstlisting}[style=customR]
print(dbGetQuery(db, "select version()"))
\end{lstlisting}
\texttt{PostgreSQL} $\textcolor{red}{\bm{11}}$\texttt{.15 ...}
\end{frame}

\begin{frame}[fragile]\frametitle{\bf\textcolor{blue}{Hands-on with Jupyterhub}}
\begin{itemize}
\item For those on Biostatistics Cluster servers gouda or colby\\
you can access Jupyterhub over the LAN
\item Many advantages using the 
tools that you use on a daily basis 
\item Here, I'm using emacs
\end{itemize}
\lstinputlisting[style=customR]{snippet3.R}
% \begin{verbatim}
% library(DBI)
% library(RPostgres)
% user="JHUSERNAME" 
% password="JHPASSWORD"
% ## db object to facilitate database two-way communication
% db=dbConnect(
%     Postgres(),           
%     dbname  ="fh_jupyter_hub_hbdb",
%     user    =user,
%     password=password,
%     host    ="garth.ctsi.mcw.edu", ## LAN connection 
%     port    =5432, 
%     bigint  ="integer"    
% )
% \end{verbatim}
\end{frame}

\begin{frame}[fragile]\frametitle{\bf\textcolor{blue}{Hands-on with Jupyterhub}}
\begin{itemize}
\item Let's see the public database tables
\end{itemize}
\lstinputlisting[style=customR]{snippet4.R}
% \begin{verbatim}
% ## ' these are single quotes 
% tables=dbGetQuery(db, 
%              "select * 
%               from information_schema.tables
%               where table_schema = 'public'")
% print(tables$table_name)
% \end{verbatim}
\end{frame}

\begin{frame}[fragile]\frametitle{\bf\textcolor{blue}{Hands-on with Jupyterhub}}
\begin{itemize}
\item Let's see the public database table columns
\end{itemize}
\lstinputlisting[style=customR]{snippet5.R}
% \begin{verbatim}
% ## ' these are single quotes 
% columns=dbGetQuery(db, 
%              "select * 
%               from information_schema.columns
%               where table_schema = 'public'")
% str(columns)
% table(columns$table_name)
% \end{verbatim}
\end{frame}

\begin{frame}[fragile]\frametitle{\bf\textcolor{blue}{Hands-on with Jupyterhub}}
\begin{itemize}
\item We have created the \texttt{columns} \texttt{data.frame}
\item We used the \texttt{str} function to see its {\it structure}: \texttt{str(columns)}
\item The first four variables look like so
\end{itemize}
\begin{verbatim}
'data.frame':	671 obs. of  44 variables:
$ table_catalog : chr "fh_jupyter_hub_hbdb" ...
$ table_schema  : chr "public" "public" "public" ...
$ table_name    : chr "fh_hb_diagnosis_jupyter" ...
$ column_name   : chr "patient_hash" "encounter_hash" ...
...
\end{verbatim}
\end{frame}


\begin{frame}[fragile]\frametitle{\bf\textcolor{blue}{Hands-on with Jupyterhub}}
\begin{itemize}
\item We can quickly summarize the tables in the database
\item \texttt{table(columns\$table\_name)}
\item For a selection, counts correspond to the number of columns as we have seen from the structure of the \texttt{data.frame}
%\item Here's an edited version focusing on the most important tables
\end{itemize}
\begin{verbatim}
      fh_hb_demographics_jupyter          
                              31
         fh_hb_diagnosis_jupyter 
                              15 
fh_hb_diagnostic_results_jupyter
                              27
         fh_hb_mar_table_jupyter
                              37
  fh_hb_med_orders_table_jupyter 
                              32 
             fh_hb_procs_jupyter
                              18 
            fh_hb_vitals_jupyter 
                              11 
\end{verbatim}
\end{frame}


\begin{frame}[fragile]\frametitle{\bf\textcolor{blue}{Hands-on with Jupyterhub}}
%\begin{center}
\begin{tabular}{lll}
Table Name \texttt{fh\_hb\_NAME\_jupyter} & 
\textcolor{PineGreen}{[\href{https://ctri.mcw.edu/wp-content/uploads/CTSI-Honest-Broker-Data-Dictionary.pdf}{Title in Documentation}]} \\ \hline
\texttt{demographics}       & ``Patient Demographics'' \\ 
\multicolumn{3}{l}{One record per patient: birth date, gender, 
race/ethnicity, death, etc.} \\ \hline 
\texttt{diagnosis}          & ``Diagnosis (Dx)'' \\
\multicolumn{3}{l}{Combo of EPIC/billing with ICD-9/ICD-10 
diagnosis codes} \\  \hline  
\texttt{diagnostic\_results}& ``Diagnostic Results'' \\
\multicolumn{3}{l}{Combo of mainly EPIC with MOSAIQ (for radiotherapy dosage)} \\  \hline  
\texttt{mar\_table}         & ``Medications Administered''\\
\multicolumn{3}{l}{EPIC \textcolor{red}{in-patient} records with Medi-Span pharm class/sub-class} \\   \hline 
\texttt{med\_orders\_table} & ``Medication Orders'' \\
\multicolumn{3}{l}{EPIC prescription orders (not fills!) along with Medi-Span} \\  \hline  
\texttt{procs}              & ``Procedures (Px)'' \\
\multicolumn{3}{l}{Combo of EPIC/billing with 
\textcolor{red}{ICD-9/ICD-10} 
and \textcolor{blue}{HCPCS/CPT codes}} \\  \hline  
\texttt{vitals}             & ``Vitals'' \\
\multicolumn{3}{l}{EPIC vital signs such as height/weight, blood pressure,
temp, etc. } \\   \hline 
\end{tabular}
%\end{center}
\end{frame}


\begin{frame}[fragile]\frametitle{\bf\textcolor{blue}{Hands-on with Jupyterhub}}
%\begin{center}
\begin{tabular}{lll}
\multicolumn{2}{c}{Froedtert Only} \\
Table Name \texttt{fh\_hb\_NAME\_jupyter} & 
\textcolor{PineGreen}{[\href{https://ctri.mcw.edu/wp-content/uploads/CTSI-Honest-Broker-Data-Dictionary.pdf}{Title in Documentation}]} \\ \hline
\texttt{naaccr}       & ``NAACCR Data'' \\ 
\multicolumn{3}{l}{North American Association of Central Cancer Registries} \\ \hline 
\texttt{surgical\_case}       & ``Surgical Case'' \\ 
\multicolumn{3}{l}{Including anesthesia, \texttt{asa\_rating\_c}, surgical service, etc.} \\ \hline
\multicolumn{2}{l}{\texttt{ekg\_patient\_tracings}, \texttt{ekg\_order},} \\
\multicolumn{2}{l}{\texttt{ekg\_test\_demographics}, \texttt{ekg\_resting\_ecg\_meas},} \\
\multicolumn{2}{l}{\texttt{ekg\_qrs\_times\_types}, \texttt{ekg\_waveform},} \\
\multicolumn{2}{l}{\texttt{ekg\_lead\_data}, \texttt{ekg\_pharma\_data},} \\
\multicolumn{2}{l}{\texttt{ekg\_meas\_matrix} \& \texttt{ekg\_meas\_matrix\_leads}} \\
           %     %ekg_lead_data                  %ekg_meas_matrix 
           %                    18                               16 
           % %ekg_meas_matrix_leads                        %ekg_order 
           %                    56                               13 
           %  %ekg_patient_tracings                  %ekg_pharma_data 
           %                     5                                6 
           %   %ekg_qrs_times_types             %ekg_resting_ecg_meas 
           %                     7                               22 
           % %ekg_test_demographics                     %ekg_waveform 
           %                    14                               11 
\multicolumn{3}{l}{GE Healthcare's MUSE for electrocardiograms: see Appendix A} \\ \hline
\end{tabular}

\end{frame}

\begin{frame}[fragile]
\textcolor{PineGreen}{[\href{https://www.ncbi.nlm.nih.gov/books/NBK441940}
  {American Society of Anesthesiologists (ASA)\\ 
physical status classification}]}
\begin{tabular}{r|l}\frametitle{\bf\textcolor{blue}{
American Society of Anesthesiologists (ASA) rating: \texttt{asa\_rating\_c}}}
Value & Definition \\ \hline
1 & healthy \\
2 & mild systemic disease \\
3 & severe systemic disease that is not life-threatening \\
4 & severe systemic disease that is a {\bf constant threat to life} \\
5 & moribund patient not expected to survive without surgery \\
6 & brain-dead patient who is an organ donor 
\end{tabular}
\end{frame}

\begin{frame}[fragile]\frametitle{\bf\textcolor{blue}{Hands-on with Jupyterhub}}
\begin{itemize}
\item When does the NAACCR data start?
\item Above, I said 1989: how did I get that?
\item N.B. Do not confuse the SQL \texttt{date\_part} function with 
the SAS \texttt{datepart} function: they are similar, yet quite different
\end{itemize}
\lstinputlisting[style=customR]{snippet6.R}
% \begin{verbatim}
% naaccr=dbGetQuery(db, 
%           "select
%            date_part('year', date_of_diagnosis_shifted)
%            as dxyear from fh_hb_naaccr_jupyter")
% t(table(naaccr$dxyear))
% \end{verbatim}
\end{frame}

\begin{frame}[fragile]\frametitle{\bf\textcolor{blue}{Hands-on with Jupyterhub}}
\begin{itemize}
\item When does the EKG data start?
\item Covered in the Appendix A of the 
\item \textcolor{PineGreen}{[\href{https://ctri.mcw.edu/wp-content/uploads/CTSI-Honest-Broker-Data-Dictionary.pdf}{CTSI Honest Broker Data Dictionary}]}
\end{itemize}
\lstinputlisting[style=customR]{snippet7.R}
% \begin{verbatim}
% ## ' these are single quotes
% muse=dbGetQuery(db, 
%           "select
%            date_part('year',acquisition_date_shifted)
%            as ekgyear from ekg_test_demographics")
% t(table(muse$ekgyear))
% \end{verbatim}
\end{frame}

\begin{frame}[fragile]\frametitle{\bf\textcolor{blue}{Hands-on with Jupyterhub}}
\begin{itemize}
\item The state of Wisconsin (like regions of the US/Canada) 
has/have a cancer registry that health care systems must
submit their incidence data to by statuary requirements
\item \textcolor{PineGreen}{[\href{https://www.naaccr.org/data-standards-data-dictionary}
{NAACCR data dictionaries}]}
\item 
\textcolor{PineGreen}{[\href{http://datadictionary.naaccr.org/default.aspx?c=10&Version=22}
{Latest version 22}]}
\item NAACCR item 523: ICD-O-3 behavior codes \texttt{behavior\_code\_icd\_o\_3}
\item ``valid codes are 0-3'' is what NAACCR specifies in their docs\\ 
which leaves a lot to be desired
\item So use the Surveillance, Epidemiology
and End Results (SEER) program docs too 
\item NCI SEER is a subset of NAACCR
regions in the US (including WI) with more carefully curated data\\
and much better documentation too!
\item \textcolor{PineGreen}{[\href{https://seer.cancer.gov/manuals/2022/SPCSM_2022_MainDoc.pdf}
{SEER Program Coding and Staging Manual}]}
\item 0: benign, 1: uncertain, 2: in situ, 3: malignant
\end{itemize}
\end{frame}

\begin{frame}[fragile]\frametitle{\bf\textcolor{blue}{Hands-on with Jupyterhub}}
\begin{itemize}
\item NAACCR item 400: Primary Site \texttt{site\_primary}
\item ICD-O-3 topography code, e.g., breast cancer C50
\item NAACCR item 560: \texttt{sequence\_number\_hospital}
\item Indicates the sequence of all malignant and nonmalignant
neoplasms over the lifetime of the patient
\item How many patients had their first breast cancer between
2016 and 2017?
\item The PostgreSQL manual seems to be very terse
\item But SAS has a nice description in their manual: \\
\textcolor{PineGreen}{[\href{https://documentation.sas.com/doc/en/pgmsascdc/9.4_3.5/sqlproc/titlepage.htm}{SAS 9.4 SQL Procedure User’s Guide, Fourth Edition}]}
\item See the \texttt{BETWEEN} and \texttt{LIKE} operators
\end{itemize}
\end{frame}

\begin{frame}[fragile]\frametitle{\bf\textcolor{blue}{Hands-on with Jupyterhub}}
\lstinputlisting[style=customR]{snippet8.R}
\end{frame}

\begin{frame}[fragile]\frametitle{\bf\textcolor{blue}{Hands-on with Jupyterhub}}
\begin{itemize}
\item But, how many of these are malignant?
\item Notice that except for the dates/times,
everything is character
\item In both R and SAS, character is more painful than numeric
values (SAS imports dates from CSV files as numeric)
\item We can export the data as a CSV with R's \texttt{write.csv}
\item The option \texttt{quote=FALSE} means don't use quotes
\item However, if some fields have embedded commas, then you
can't turn this option on: and we don't know if there are any
\item The \texttt{na} option determines how \texttt{NA} is handled
\item In this case, we set it to the CSV standard: \texttt{na=""}
for nothing 
\end{itemize}
\lstinputlisting[style=customR]{snippet9.R}
% \begin{verbatim}
% addmargins(table(naaccr$site_primary, 
%            naaccr$behavior_code_icd_o_3))
% write.csv(naaccr, "naaccr.csv", row.names=FALSE, na="") 
% str(naaccr)
% ?str
% \end{verbatim}
\end{frame}



\begin{frame}[fragile]\frametitle{\bf\textcolor{blue}{Honest Broker Residential
Address Geocoding}}
\begin{itemize}
\item Geocoding is the process of taking the address of a location and
  returning geographic coordinates\\
 such as a latitude/longitude pair or a US Census Tract
\item  
\textcolor{PineGreen}{[\href{https://degauss.org}{DeGAUSS:
    Decentralized Geomarker Assessment for Multi-Site Studies}]}
\item ``Patient Demographics'' \texttt{geocode\_result}
\item Self-explanatory codes that are not accurately geocoded:\\ 
\texttt{po\_box} and \texttt{non\_address\_text} 
\item \texttt{cincy\_inst\_foster\_addr}: the address was not geocoded \\ 
a known institutional address, not a residential address
\item \texttt{imprecise\_geocode}: the address was geocoded, but results were suppressed because due to the lack of precision
\item \texttt{geocoded}: the address was geocoded with precision
\end{itemize}
\end{frame}

\begin{frame}[fragile]\frametitle{\bf\textcolor{blue}{Introduction to
the Area Deprivation Index (ADI)}}
\begin{itemize}
\item \textcolor{PineGreen}{[\href{https://www.neighborhoodatlas.medicine.wisc.edu}{ADI documentation}]}
\item ``Patient Demographics'' \texttt{adi\_narank}
\item A Census Block Group (CBG) with an ADI ranking of 1 indicates the lowest level of disadvantage within the nation, i.e., least deprived 1st percentile
\item A CBG with an ADI ranking of 100 indicates the highest level of disadvantage, i.e., most deprived 100th percentile
\item So the range is integers 1 to 100
\item Some CBG are missing ADI rankings with the following codes
\item \texttt{NA} the R missing value code which is undocumented
\item \texttt{'PH'} for suppression due to low population and/or housing 
\item \texttt{'GQ'} for suppression due to a high group quarters population
\item \texttt{'PH-GQ'}  (or \texttt{'GQ-PH'}?) 
for suppression due to both types
\item \texttt{'KVM'} (or \texttt{'QDI'}?) 
designates block groups without an ADI value due to 
  missing data within the data's source\\
the American Community Survey Five Year Estimates
\end{itemize}
\end{frame}

\begin{frame}[fragile]\frametitle{\bf\textcolor{blue}{Hands-on with Jupyterhub}}
\lstinputlisting[style=customR]{snippet10.R}
% \begin{verbatim}
% a=dbGetQuery(db, "select geocode_result 
%                   from fh_hb_demographics_jupyter")
% table(a$geocode_result, useNA='ifany')
% b=dbGetQuery(db, "select adi_narank 
%                   from fh_hb_demographics_jupyter
%                   where geocode_result='geocoded'")
% table(b$adi_narank, useNA='ifany')
% unk = (is.na(b$adi_narank) | 
%        b$adi_narank %in% c('GQ', 'PH', 'GQ-PH', 'QDI'))
% table(unk)
% b$adi=0
% b$adi[unk]=NA
% b$adi[!unk]=as.integer(b$adi_narank[!unk])
% summary(b$adi)
% plot(density(b$adi, from=1, to=100, na.rm=TRUE), 
%      xlab='ADI', ylab='Distribution', 
%      main='', sub='1:least, 100:most')
% abline(h=0.01, col=8)
% \end{verbatim}
\end{frame}

\begin{frame}[fragile]\frametitle{\bf\textcolor{blue}{Death and Cause
      of Death}}

\begin{itemize}
\item Death is available; however, it may not be trustworthy
\item It is common to see EHR data far beyond the death date
\item Also, Cause of Death is not available
\item It can be acquired from the state of Wisconsin or\\
the US Centers for Disase Control and Prevention (CDC) \\ 
National Death Index (NDI)
\item NDI is death only; NDI Plus includes cause % and costs more
\item Of course, this costs money and cause costs more
\item Typically, NDI/NDI Plus is one to two years behind
\item Complete data for 2020 was released in 01/2022
\item Early release for 2021 in 01/2022 was about 98\% complete 
\item And the caveats about cause of death are substantial
\item \textcolor{PineGreen}{[\href{https://www.cdc.gov/nchs/ndi/index.htm}
{US CDC NDI}]}
\end{itemize}

\end{frame}


\begin{frame}[fragile]\frametitle{\bf\textcolor{blue}{Anomylous Death Dates}}
\begin{itemize}
\item There appears to be an issue with the status of death\\
and/or the death dates in the ``Patient Demographics'' table
\item It is all relative, but the size of the ``Vitals'' table is much smaller
than other tables such as ``Encounters''
\item Presumably, there would not be vital signs recorded on a date after someone has died
\item So, we can create a relatively undemanding check of death dates
by comparisons with their vital signs 
\end{itemize}
\end{frame}


\begin{frame}[fragile]\frametitle{\bf\textcolor{blue}{Hands-on with Jupyterhub}}
\lstinputlisting[style=customR]{snippet11.R}
\end{frame}


\begin{frame}[fragile]\frametitle{\bf\textcolor{blue}{Hands-on with Jupyterhub}}
Let's compare the last date of vital signs with death date
\lstinputlisting[style=customR]{snippet12.R}
% \begin{verbatim}
% ## ' these are single quotes 
% in.clause=paste0("('", paste(zombies$patient_hash, 
%                  collapse="','"), "')")
% zdates=dbGetQuery(db, 
%           paste("select patient_hash, measure_date_shifted 
%                  from fh_hb_vitals_jupyter  
%                  where patient_hash in ", in.clause, 
%                 "order by patient_hash, 
%                  measure_date_shifted"))           
% max.zdates=c(tapply(zdates$measure_date_shifted, 
%                     zdates$patient_hash, max))
% class(max.zdates)='POSIXct'
% print(cbind(format(zombies$death_date_shifted), 
%             format(max.zdates)))
% \end{verbatim}
\end{frame}

\begin{frame}[fragile]\frametitle{\bf\textcolor{blue}{Medi-Span, GPI and RxNorm
Medical Nomenclature}}
\begin{itemize}
\item \textcolor{PineGreen}{[\href{https://www.wolterskluwer.com/en/solutions/medi-span/about/gpi}
{Medi-Span Generic Product Identifier (GPI)}]}
\item The Wolters Kluwer Medi-Span brand database, called the Medispan
  Electronic Drug File, links the GPI code to other prescription drug
  classification codes
\item \textcolor{PineGreen}{[\href{https://www.nlm.nih.gov/research/umls/rxnorm/index.html}{RxNorm}]} is part of Unified Medical Language System (UMLS)
  terminology maintained by the US National Library
  of Medicine (NLM)
\item GPI and RxNorm codes are available on two CRDW tables
\item ``Medication Orders'' for medicinal prescriptions (not fills!): \texttt{fh\_hb\_med\_orders\_table\_jupyter}
\item ``Medications Administered'' for medicine given: \texttt{fh\_hb\_mar\_table\_jupyter} 
\item Example variables of interest
\item \texttt{pharm\_class}: pharmacologic class
\item \texttt{pharm\_subclass}: pharmacologic subclass
\item \texttt{ingredient\_rxcui\_name}: RxNorm Concept Unique Identifier (CUI) name
\end{itemize}
\end{frame}

\begin{frame}[fragile]\frametitle{\bf\textcolor{blue}{Hands-on with Jupyterhub}}
How to make your own lookup table of drug nomenclature\\ 
N.B. RESOURCE INTENSIVE: JUST DON'T DO IT TODAY
\lstinputlisting[style=customR]{snippet13.R}
% \begin{verbatim}
% medispan=dbGetQuery(db, 
%              "select distinct pharm_class, pharm_subclass, 
%               substring(gpi from 1 for 2) as gpi_group,
%               substring(gpi from 3 for 2) as gpi_class,
%               substring(gpi from 5 for 2) as gpi_subclass,
%               ingredient_rxcui_name
%               from fh_hb_mar_table_jupyter 
%               order by pharm_class, pharm_subclass,
%               ingredient_rxcui_name")
% write.csv(medispan, "medispan.csv", row.names=FALSE, na="")
% \end{verbatim}
\end{frame}

\begin{comment}
\begin{frame}[fragile]\frametitle{\bf\textcolor{blue}{Cancer Research with RASMACRO}}

\begin{itemize}
\item We have already mentioned NCI SEER
\item Another resource is NCI's
\textcolor{PineGreen}{[\href{https://healthcaredelivery.cancer.gov/seermedicare}
{SEER-Medicare Linked Database}]}
\item For example, \textcolor{PineGreen}{[\href{https://healthcaredelivery.cancer.gov/seermedicare/aboutdata/hcpcs.html}{a resource for identifying HCPCS/CPT codes}]}
%\item Such as \textcolor{PineGreen}{[\href{https://healthcaredelivery.cancer.gov/seermedicare/aboutdata/ndc_frequency.html}{oral chemotherapy drug names}]}
\item

\item Here, automated processing would be very challenging with R
\item This is a job for ... RASMACRO!
\item See the file \texttt{oral\_meds.sas}
\end{itemize}
\end{frame}
\end{comment}


\begin{frame}[fragile]\frametitle{\bf\textcolor{blue}{Hands-on with Jupyterhub}}
What types of procedure codes do we have? \\ 
N.B. RESOURCE INTENSIVE: JUST DON'T DO IT TODAY
\lstinputlisting[style=customR]{snippet14.R}
\end{frame}


\begin{frame}[fragile]\frametitle{\bf\textcolor{blue}{Hands-on with Jupyterhub}}
\begin{itemize}
\item What types of procedure codes do we have in the EHR? \\ 
\item The Centers for Medicare and Medicaid Services (CMS)
was an early entrant into the electronic billing space
\item The Healthcare Common Procedural Coding System (HCPCS)
was created by CMS for billing purposes
\item HCPCS Level I codes are the \textcolor{PineGreen}{[\href{https://www.ama-assn.org/amaone/cpt-current-procedural-terminology}{American
Medical Association's Common Procedural Terminology
fourth edition (CPT-4)}]} that typically have 5 digits
(or occasionally 4 digits followed by a letter: see \texttt{snippet15.R}) 
\item \textcolor{PineGreen}{[\href{https://www.cms.gov/Medicare/Coding/HCPCSReleaseCodeSets/Alpha-Numeric-HCPCS}{HCPCS Level II}]} codes start with a letter 
typically followed by 4 digits (see \texttt{snippet15.R})
\end{itemize}

\begin{tabular}{llr}
\texttt{px\_type} & Coding system & Percentage \\
\texttt{09} & ICD-9  & 0.1\% \\
\texttt{10} & ICD-10 & 0.3\% \\
\texttt{CH} & HCPCS/CPT & \textcolor{red}{88.0\%} \\
\texttt{OT} & Custom & 11.0\% \\
\texttt{NA} & Not available & 0.7\%
\end{tabular}
\end{frame}

\begin{frame}[fragile]\frametitle{\bf\textcolor{blue}{Cancer Research with RASMACRO}}

\begin{itemize}
\item We have already mentioned NCI SEER
\item Another resource is NCI's
\textcolor{PineGreen}{[\href{https://healthcaredelivery.cancer.gov/seermedicare}
{SEER-Medicare Linked Database}]}
\item For example, \textcolor{PineGreen}{[\href{https://healthcaredelivery.cancer.gov/seermedicare/aboutdata/hcpcs.html}{a resource for identifying CPT/HCPCS codes}]}
%\item \textcolor{PineGreen}{[\href{https://www.cms.gov/Medicare/Coding/HCPCSReleaseCodeSets/Alpha-Numeric-HCPCS}{List of HCPCS codes with descriptions}]}
%\item Such as \textcolor{PineGreen}{[\href{https://healthcaredelivery.cancer.gov/seermedicare/aboutdata/ndc_frequency.html}{oral chemotherapy drug names}]}
\item   \textcolor{PineGreen}{[\href{https://healthcaredelivery.cancer.gov/seermedicare/aboutdata/breast.all.hcpcs.table.txt}{See
      this badly formatted list of breast cancer patient codes}]}
\item Here, automated processing would be very challenging with R
\item This is a job for ... RASMACRO!
\item See the file \texttt{breastcodes.sas}
\end{itemize}
\end{frame}


\begin{frame}[fragile]\frametitle{\bf\textcolor{blue}{Cancer Research with RASMACRO}}

\begin{itemize}
\item For example, we have an EHR study of breast cancer survivors
%\item Which \textcolor{PineGreen}{[\href{https://www.ama-assn.org/amaone/cpt-current-procedural-terminology}{AMA Common Procedural Terminology (CPT) codes}]}
\item Which codes are for breast cancer treatment and which standard care?
\item We need non-breast cancer men and women to compare with
\item Let's contrast HCPCS codes with colorectal cancer patients
\item What is left over are very likely breast cancer treatment
\item Let's examine the codes billed to Medicare part A/B in 2016
% \item We can look these up in the AMA CPT books
% that are available in the MCW library
\end{itemize}
\end{frame}


\begin{frame}[fragile]\frametitle{\bf\textcolor{blue}{Cancer Research with RASMACRO}}

\begin{center}
\begin{tabular}{rr|l}
CPT & Bills & Description \\ \hline
19380 & 1690 & revision of reconstructed \textcolor{red}{breast} \\
19357 & 1448 & \textcolor{red}{breast} reconstruction \\
19340 & 1281 & immediate insertion of \textcolor{red}{breast} prosthesis \\
11970 & 1062 & replacement of tissue expander with prosthesis \\
19371 & 1031 & periprosthetic caspulectomy of \textcolor{red}{breast} \\
19342 & 1006 & delayed insertion of \textcolor{red}{breast} prosthesis \\
19316 & 788 & mastopexy \\
19283 & 717 & placement of \textcolor{red}{breast} localization device \\
19328 & 660 & removal of intact mammary implant \\
19350 & 638 & nipple/areola reconstruction \\
19370 & 599 & open periprosthetic caspulectomy of \textcolor{red}{breast} \\
93702 & 599 & bioimpedance spectroscopy analysis for lymphedema \\
19499 & 520 & unlisted \textcolor{red}{breast} procedure \\
77058 & 507 & magnetic resonance imaging of the \textcolor{red}{breast} \\
19286 & 424 & placement of \textcolor{red}{breast} localization device \\
81212 & 419 & procedure for 185delAG, 5385insC, 6174delT variants 
\end{tabular}
\end{center}

\end{frame}

\begin{frame}[fragile]\frametitle{\bf\textcolor{blue}{Cancer Research with RASMACRO}}

\begin{center}
\begin{tabular}{rr|l}
CPT & Bills & Description \\ \hline
19296 & 388 & placement of radiotherapy catheter into the \textcolor{red}{breast} \\
11921 & 369 & tattooing; 6.1 to 20 sq cm \\
38740 & 325 & axillary lymphedectomy; superficial \\
19297 & 316 & radiotherapy catether placement with partial masectomy \\
38790 & 295 & lymphangiography injection \\
19086 & 279 & \textcolor{red}{breast} biopsy of additional lesion \\
77469 & 273 & intraoperative radiation treatment management \\
19126 & 252 & excision of \textcolor{red}{breast} lesion identified by 
radiological marker \\
77424 & 252 & intraoperative radiation delivery, single x-ray treatment \\
19361 & 239 & \textcolor{red}{breast} reconstruction with latissimus dorsi flap \\
11971 & 223 & removal of tissue expanders without prosthesis insertion \\
19330 & 223 & removal of mammary implant material \\
19364 & 213 & \textcolor{red}{breast} reconstruction with free flap \\
19325 & 179 & mammaplasty augmentation with prosthetic implant \\
64462 & 174 & paravertebral block injection including imaging guidance \\
11922 & 163 & tattooing; each additional 20 sq cm
\end{tabular}
\end{center}

\end{frame}

\begin{frame}[fragile]\frametitle{\bf\textcolor{blue}{Cancer Research with RASMACRO}}

\begin{center}
\begin{tabular}{lr|l}
HCPCS & Bills & Description \\ \hline
L8000 &      816  &     Mastectomy bra                  \\
J9179 &      705  &     Eribulin mesylate injection     \\
J9354 &      693  &    Inj, ado-trastuzumab emt 1mg    \\
Q4116 &      427  &    Alloderm                        \\
L8600 &      335  &    Implant \textcolor{red}{breast} silicone/eq      \\
L8030 &      330  &    \textcolor{red}{Breast} prosthes w/o adhesive    \\
C1728 &      247  &    Cath, brachytx seed adm         \\
C8906 &      245  &    Mri w/cont, \textcolor{red}{breast},  bi         \\
C8905 &      239  &    Mri w/o fol w/cont, \textcolor{red}{breast}, un    \\
C8907 &      171  &    Mri w/o cont, \textcolor{red}{breast}, bi        \\
G8875 &      171  &    \textcolor{red}{Breast} cancer dx min invsive    \\
J9207 &      134  &    Ixabepilone injection           \\
C9726 &      116  &    Rxt \textcolor{red}{breast} appl place/remov     \\
C9728 &      103  &    Place device/marker, non pro    \\
J1950 &       88  &    Leuprolide acetate /3.75 mg     \\
G8872 &       85  &    Intraop image confirm excise    \\
C2639 &       72  &    Brachytx, non-stranded,i-125    
\end{tabular}
\end{center}

\end{frame}

\begin{frame}[fragile]\frametitle{\bf\textcolor{blue}{Hands-on with Jupyterhub}}

\begin{itemize}
\item Froedtert is a Level 1 trauma center so it has a trauma registry
\item However, you need IRB approval to utilize the trauma registry
\item But you can get some of the data from \texttt{surgical\_case}
\item How many surgical trauma \textcolor{red}{patients} are there per year?
%\end{itemize}
%\end{frame}

%\begin{frame}[fragile]\frametitle{\bf\textcolor{blue}{Hands-on with Jupyterhub}}
\item Here's the number of trauma surgeries performed in 2018
\end{itemize}
\lstinputlisting[style=customR]{snippet16.R}
\end{frame}


\begin{frame}[fragile]\frametitle{\bf\textcolor{blue}{Hands-on with Jupyterhub}}
\begin{itemize}
\item But we want the number of trauma \textcolor{red}{patients} undergoing surgery in 2018
\item We can use the \texttt{byvalue} function to identify the first
patient record by date
\item You have to upload the \texttt{byvalue.R} file to Jupyterhub
\item How many surgical trauma \textcolor{red}{patients} are there per year?
\end{itemize}
\lstinputlisting[style=customR]{snippet17.R}
\end{frame}

\begin{frame}[fragile]\frametitle{\bf\textcolor{blue}{Hands-on with Jupyterhub}}
\begin{itemize}
\item What about their trauma diagnoses?
\item We can use the \texttt{inclause} function to create an \texttt{IN}
clause to merge with the \texttt{diagnosis} table
\item You have to upload the \texttt{inclause.R} file to Jupyterhub
\item How many surgical trauma \textcolor{red}{patients} 
first encounters linked to the \texttt{diagnosis} table?
\end{itemize}
\lstinputlisting[style=customR]{snippet18.R}
\end{frame}

\end{document}

